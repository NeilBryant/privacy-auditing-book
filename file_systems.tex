\chapter{File System Analysis and File Extraction}

\sgraphic{art/filesystems}{Mass storage devices are typically used to
  hold one or more file systems. Disks start at Logical Block Address
  0. Legacy partitioning schemes such as Master Boot Record (MBR)
  start at LBA 0, while the GUID Partitioning Table (GPT) schem starts
  at LBA 1. In either case the partition table conveys the number of
  partitions and the address at which each one starts. There may be
  gaps between the partition map and the first file systems, and there
  may be gaps between the file systems. These gaps may be blank or may
  contain residual data from previous uses of the disk---ghost file
  systems from an early
  time.
}

In the previous chapter we discussed how data on a mass storage device
can be extracted and preserved. In this chapter we will begin the
analysis of data that has been successfully extracted from a mass
storage device.

\section{Background}
Although there are many ways to store information on a hard drive or
other mass storage device, most drives are used to store one or more
\emph{file systems}. These file systems, in turn, are used to store
named files which are themselves arranged in named directories. 

\subsection{Partitioning and Volume Management}

Although a file system can be stored directly on the drive, with the
file sector of the drive being used to store the first sector of the
file system, this is not commonly done. Instead, the first 1 or 2
sectors are used for a \emph{partition table} that describes the
locations of the file systems. In this way a single device can be used
to simulate multiple logical devices. Two commonly used partitioning
schemes are the \emph{Master Boot Record} and the \emph{GUID Partition
  Table} scheme.\wikipedia{http://en.wikipedia.org/wiki/GUID_Partition_Table}
       \wikipediab{http://en.wikipedia.org/wiki/Master_boot_record}

The term \emph{physical volume} is used variously to describe a
physical disk, a partition on a physical disk, or a Logical Unit
Number (LUN) of a storage
system.\wikipedia{http://en.wikipedia.org/wiki/Logical_Unit_Number}
Physical volumes can be used directly to hold a file system or can be
grouped together with a Logical Volume Management (LVM)
system\wikipedia{http://en.wikipedia.org/wiki/Logical_volume_management}. An
LVM can group multiple physical volumes together in to a single
physical volume that is large or more reliable than the volumes that
it built from. Volume managers can also implement
\emph{snapshots}\wikipedia{http://en.wikipedia.org/wiki/Snapshot_(computer_storage)}
through the use of
\emph{copy-on-write}\wikipedia{http://en.wikipedia.org/wiki/Copy-on-write}.

\subsection{Files and File Systems}
The word \emph{file} is commonly used to describe a sequence of data
that is stored on a computer mass storage system. On modern computers
files are sequences of zero or more \emph{bytes}. Files have a length
that is set at any given point in time, but can be readily
changed. Files can have names and other associated \emph{metadata}
such as timestamps, an owner, and a group. Files have traditionally
been grouped together in
\emph{directories}\wikipedia{http://en.wikipedia.org/wiki/File_directory}. also
called \emph{folders}. 

The phrase \emph{File
  System}\wikipedia{http://en.wikipedia.org/wiki/Computer_file} is
used to describe the part of the computer's operating
system that manages a storage system. However, the term is also used
to describe the very sectors on a mass storage device that contain the
files.

It is important to realize that files are an \emph{abstraction} that
was created for the purpose of managing data. There is nothing
inherent in the design of computers, operating systems or mass storage
devices that requires the use of files. A raw disk can be used as
virtual memory \emph{backing store} or as a storage system for a
database. Instead of stored as a sequence of bytes in a file, data can be stored as
objects that are persisted in memory, or as BLOBs in a database.

It is also important to realize that any computational storage device
or abstraction can be used to virtualize and contain any other storage
device. That is, a file can be stored in a file system, but a file
system can also be created and stored inside a file (this is
essentially what a disk image is). Most operating systems can swap to raw
partitions or to files, but file systems can also be created inside
RAM or virtual memory.

Adding to the complexity of the forensic examiner is
\emph{encryption}, which can be applied at the interface between any
of these abstractions. Encryption can be performed in the storage
device itself (as in the case of an encrypting hard drive), in the
disk driver, in the logical volume management system, in the file
system, or at the application level.

\subsection{File Systems of Forensic Interest}
There are a variety of different kinds of file systems in use on
modern computer systems:
\begin{description}
\item[Disk file systems] organize files and directories on
  block-oriented storage systems. These are of interest to those
  engaged in MEDEX operations. Popular file systems include FAT32
  (used primary on removable storage devices and camera cards), NTFS
  (Microsoft's New Technology File System), and HFS+ (Apple's
  Hierarchical File System used on Macs and iPhones).
\item[Distributed file systems] allow a computer to access
  information on remote servers as if it is stored
  locally. Distributed file systems are forensically interesting
  because many use local storage to cache information from the remote
  servers. Analyzing local storage can therefore give clues as to what
  was accessed remotely, and when it was accessed.
\item[Virtual file systems] use the file and directory abstraction to
  make it easy to access other information. For example, the Linux
  \url{/dev} file system is used to access devices through the file
  system, the \url{/sys} file system is used to access features within
  the system kernel, and the \url{/net} file system accesses the
  automounter. Virtual file systems are not typically of interest in
  MEDEX because they do not leave residual information on a storage
  device. However, virtual file systems are relevant in malware
  analysis and intrusion response.
\end{description}

As this chapter is written, specific file systems of interest include:
\begin{description}
\item[FAT12, FAT16 and FAT32] file systems developed by Microsoft for use with
  DOS. FAT refers to the File Allocation Table, an array of integers
  that is used to determine if a cluster is in use or free. (In FAT, a
  \emph{cluster} is a block of 1, 2, 4, 8 or more disk sectors.) The number
  refers to the size of the integers in the array (12 bits, 16 bits or
  32 bits). Today implementations of FAT are built in to practically every
  operating system, including Windows, Linux, MacOS, most digital
  cameras, and practically every other device with a USB or SD
  interface. 
\item[NTFS] 
\item[HFS+]
\item[YAFFS2]
\item[EXT2/3]
\item[EXT4]
\end{description}

Information can be hidden in a file system by storing data in blocks
that are allocated but not used to hold content\cite{dfrws2005:KnutEcksteinAndMarkoJahnke}. 

\subsection{File System Structures and Terminology}

Blocks

Clusters

inode

Hard Link

Symbolic Link

Master File Table

File Allocation Table

Free List

Unlink

\subsection{Classification of file system information}

Allocated

Deleted

 - Recoverable through undeletion

 - Recoverable through carving

Partially overwritten

\subsection{SSDs and the ``TRIM'' command}
\cite{dfrws2011:TimothyVidasAndChengyeZhangAndNicolasChristin}


\section{File Deletion and Deleted File Recovery}\label{deleted_file_recovery}
Different file system implementations delete files in different ways.

Traditionally, file systems simply \emph{unlinked} files---the pointer
to the file was removed from the directory, and the blocks associated
with the file were returned to the free list.


\section{Exploiting File System Metadata}
File system metadata can be used to determine usage. 
\cite{dfrws2011:JonathanGrier}

% October 12
