\chapter{Memory Analysis}\label{ch:ram}
\subsection{Memory Parsing}

The memory of a desktop, laptop or cell phone is a mosaic of 4096-byte
blocks that variously 
contain running program code,  fragments of programs
that recently ran and have exited, portions of the computer's
operating system, fragments of what was sent and received over the
network, fragments of windows displayed on the computer's screen, the
computer's copy-and-paste buffer, and other kinds of information. Memory
changes rapidly---typical memory systems support several billion
changes per second---so it is nearly impossible to make a copy that is
internally consistent without halting the machine. An added complication is that the very
specific manner that programs store information in memory is rarely
documented and changes between one version of a program and
another. As a result, each version may need to be painstakingly reverse-engineered by
computer forensics researchers. Thus, memory analysis is time consuming, very
difficult, and necessarily incomplete.

Despite these challenges, recent years has seen the development of
forensically sound techniques for acquiring and analyzing the contents
of a running computer system. Today there are both open source and
proprietary memory analysis tools. These tools can take a memory dump
and report the system time when the memory was captured, display a
list of running processes, open files, and even display the contents
of the computer's clipboard and screen. Today memory analysis tools
are widely used for reverse-engineering computer viruses, worms and
other kinds of \emph{malware}, as well as for understanding an
attacker's actions in computer intrusion cases. Memory analysis can be combined
with carving to recover digital photographs and video.



\section{Acquiring RAM}
\section{Analyzing RAM with volatility}
