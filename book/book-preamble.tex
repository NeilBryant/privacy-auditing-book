%
% This file has various LaTeX commands that we use for the book no
% matter which style it is being printed with
%

% Something that needs to be done
\newcommand{\citet}[1]{\cite{#1}}
\newcommand{\citep}[1]{\cite{#1}}
\newcommand{\tk}[1]{\large\textbf{\emph{#1}}}
\newcommand{\slgbookauthor}{Simson L.\ Garfinkel\\Kevin D. Fairbanks\\Golden G. Richard III}
\newcommand{\slgbooktitle}{Technical Privacy Auditing with Computer Forensics Tools} 
\newcommand{\slgbookcopyright}{Public Domain}
\newcommand{\slgbookpublisher}{DRAFT}
\newcommand{\slgbookurl}{www.digitalcorpora.org}
\newcommand{\slgbookfrontmatter}{This is a work of the US
  Government. In accordance with 17 USC 105, copyright protection is
  not available for any work of the US Government.
  \\
  This text is distributed in the hope that it will be useful, but
  WITHOUT ANY WARRANTY; without even the implied warranty of
  MERCHANTABILITY or FITNESS FOR A PARTICULAR PURPOSE.}
\newcommand{\slgbookdate}{Working Draft, May 2013}


% Make spacing look better
\renewcommand{\topfraction}{0.99}
\renewcommand{\floatpagefraction}{0.99}


% http://tex.stackexchange.com/questions/27663/using-bold-italic-text-inside-listings
% \usepackage{lmodern}
\usepackage[scaled=0.90]{couriers}

\setlength{\abovecaptionskip}{18pt}
\setlength{\belowcaptionskip}{18pt}


% http://tex.stackexchange.com/questions/17009/how-to-change-code-example-font-in-listings
\usepackage{eurosym}
\usepackage{listings}           % source code listings
\usepackage{xcolor}             % better colors, needed for soul
\usepackage{soul}               % better highlighting
\usepackage{epigraph}
%\usepackage{bibentry}
\newcommand{\bibentry}[1]{}
\newcommand{\furl}[1]{\footnote{\url{#1}}}

% http://tex.stackexchange.com/questions/32457/make-all-figures-and-tables-framed-by-default
% \usepackage[framestyle=fbox,framefit=yes,heightadjust=all,framearound=all]{floatrow} 

% http://stackoverflow.com/questions/2270082/latex-how-do-i-change-the-font-size-of-a-table-column
\usepackage{array}

%%
%% BEGIN Code environment that allows (@ and @) for escaping
%%
  \lstnewenvironment{code}[1][]
    {\lstset{language=[LaTeX]TeX}\lstset{escapeinside={(@}{@)},
     breaklines=false,
     numbers=none,
         framesep=5pt,
         basicstyle=\normalsize\ttfamily,
         showstringspaces=false,
         keywordstyle=\itshape\color{blue},
         stringstyle=\color{maroon},
      commentstyle=\color{black},
      rulecolor=\color{black},
      xleftmargin=0pt,
      xrightmargin=0pt,
      aboveskip=\medskipamount,
      belowskip=\medskipamount,
             backgroundcolor=\color{white}, #1
  }}
  {}
%%
%% END
%%

\usepackage{xspace}             % optional spaces
\usepackage{tabularx}           % the ``X'' for tables
\usepackage{pifont}             % symbols
\usepackage{color}              % some people like commenting in color
\usepackage{multirow}
\usepackage{varioref}
\usepackage{fancyvrb}
  \DefineShortVerb{\|}            % makes |foo| a verbatim command


% Useful cross-reference commands
\newcommand{\Sref}[1]{\S\ref{#1}}
\newcommand{\chapref}[1]{Chapter~\ref{#1}\xspace}
\newcommand{\chapvref}[1]{Chapter~\vref{#1}\xspace}
\newcommand{\figref}[1]{Figure~\ref{#1}\xspace}
\newcommand{\lstref}[1]{Listing~\ref{#1}\xspace}
\newcommand{\secref}[1]{Section~\ref{#1}\xspace}
\newcommand{\secvref}[1]{Section~\vref{#1}\xspace}
\newcommand{\figvref}[1]{Figure~\vref{#1}\xspace}
\newcommand{\tabref}[1]{Table~\ref{#1}\xspace}
\newcommand{\tabvref}[1]{Table~\vref{#1}\xspace}
\newcommand{\appref}[1]{Appendix~\ref{#1}\xspace}
\newcommand{\appvref}[1]{Appendix~\vref{#1}\xspace}

% Useful symbols and abbreviations
%\newcommand{\checkmark}{\Pisymbol{pzd}{52}}
\newcommand{\naive}{na\"{\i}ve\xspace}
\newcommand{\etc}{\emph{etc.}\xspace}
\newcommand{\eg}{\emph{e.g.\ }}
\newcommand{\etal}{\emph{et al.\ }}

% Rights
\newcommand{\rights}[1]{\tiny}


% \sgraphic[optional width=8in]{filename}{caption}
% Then use \figref{filename} to get a reference to the figure
\newcommand{\sgraphic}[3][width=\linewidth]{
  \begin{figure}
  \begin{center}
  \fbox{\includegraphics[#1]{#2}}
  \end{center}
  \caption{#3\label{#2}}
  \end{figure}
}

% \bifigure{image1}{image2}{caption1}
\newcommand{\bifigure}[3]{
  \begin{figure}
  \includegraphics[width=.47\linewidth]{#1} \hfill \includegraphics[width=.47\linewidth]{#2} \\
  \caption{#3\label{#1}}
  \end{figure}
}



% \twofigures{width1}{image1}{caption1}
%            {width2}{image2}{caption2}
\newcommand{\twofigures}[6]{
  \begin{figure}
    \begin{tabularx}{\textwidth}{p{#1}Xp{#4}}
  \includegraphics[width=#1]{#2} &&
  \includegraphics[width=#4]{#5} \\
  \caption{#3\label{#2}} &&
  \caption{#6\label{#5}} \\
\end{tabularx}
\end{figure}
}



%%
%% Bibliographical Information
%%


