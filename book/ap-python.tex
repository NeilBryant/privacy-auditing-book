\section{Python}
This book makes extensive use of the Python programming language for
its examples. All of the examples are tested under Python~3.3,
although Python~2.7 may work with most of the demonstration programs
as well.  This section describes specific aspects of the Python
programming language that are important in our examples and that may
be missed in an introductory Python programming class.

\subsection{Reading and Writing Files: Binary vs.\ Text}

When a Python program opens a file that file may be opened in one of
two modes: \emph{text} or \emph{binary}.  When opened as text, Python
assumes that the files contain text characters that have been encoded
with an appropriate codec---for example, that they are Unicode files
coded as UTF-8. Assuming such a codec qcan cause problems when reading
binary data, since many binary combinations do not represent valid
characters. To avoid this problem, files will normally be opened in
binary mode. This is done by opening files with method
|open(filename,"rb")|.

\subsection{Python's struct.unpack}
Much of the information on computers is stored as binary encoded
values. Such values can be easily read and written using python's
|struct.unpack| method, which reads a binary structure and returns a
python tuple of the decoded values.  

The binary structure is decoded using a \emph{format
  specification}. For example, this Python fragment will print the
decimal value of the hex string DEADBEEFh:

\lstinputlisting[caption=A program to print the decimal value of the hexadecimal number \texttt{DEADBEEFh}]{ch-1/deadbeef.py}


\subsection{Path Delimiter}
Windows uses the backslash (|\|) as a path delimiter, while Unix
systems use the forward slash (|/|). In general this is not a problem,
because Python run on Windows will treat either slash as a path
delimiter. So in general, this book will use the forward slash as a
path delimiter, even for code that runs on Windows-based computers.

\subsection{Accessing Raw Devices}

A ``raw'' device is a device file that can be opened by user programs
like a normal file but reads and writes are sent directly to the
underlying physical device. Typically access to the raw disk devices 
requires administrative privileges, since access to the raw device
bypasses the computer's operating system.

To access the physical drives on a Windows computer open the hidden
physical drives |\\.\PhysicalDrive0| through
|\\.\PhysicalDriveNN|. To open the physical drive for reading use use
mode |rb|, for writing use the ``updating'' mode |rb+|. List the
physical drives on your Windows system with the command:

\begin{code}
c:\ (@ \hl{wmic diskdrive list brief /format:list} @)
\end{code}

Unix-based systems associate two pseudo-files in the |/dev/| directory with
each physical device; additional pseudo-files map to disk
partitions. Macs use |/dev/diskNN| for block devices and
|/dev/rdiskNN| for raw devices. Partitions are identified by appending
a |sJ|, where |J| is the partition number.  List the physical drives
on a Mac with the command:

\begin{code}
$ (@ \hl{ls -l /dev/disk*} @)
\end{code}
%$

Linux systems name block devices |/dev/sdA| where |A| is a letter |a|
through the highest device letter; partitions are identified with an
appended letter. Modern Linux systems also map physical devices in
additional locations, including the directory |/dev/disk/by-uuid/| for
a list of drives by UUID, |/dev/disk/by-partuuid| for partitions by
UUID, |/dev/disk/by-path| for their location on the PCI or SCSI bus,
and |/dev/disk/by-id| for device IDs. Linux systems may also map
block devices in the |/sys/dev/block/| and character devices in
|/sys/dev/char/|. List the physical drives
on a Mac with the command:

\begin{code}
$ (@ \hl{ls -l /dev/sd*} @)
\end{code}
%$


\subsection{Making Graphs}
There are a large and growing number of producing graphical displays
from Python. This book uses the Python library matplotlib for data
visualization and graphiviz.


\subsection{Python3 on Windows}
Windows users should download the most recent version of Python 3 from
\url{http://www.python.org/getit}. Use the \emph{Windows X86-64 MSI
  Installer} if you have a 64-bit system and a \emph{Windows x86 MSI
  Installer} for 32-bit system. Although the 32-bit version will work
on a 64-bit system, 32-bit programs running on a 64-bit system are
presented with a slightly incorrect view of the computer's file
system, which can be confusing in some instances.\footnote{Microsoft
  uses the term Windows on Windows 64, or WOW64, to describe the
  process of running 32-bit applications on 64 bit Windows. For a
  complete discussion, search for WOW64 on MSDN or Wikipedia.}


On Windows, install matplotlib for Python 3.3 with:

\subsection{Python3 on Linux}
On Linux systems you may need to explicitly install Python 3. Do so on
Fedora by typing:

\begin{code}
$ (@ \hl{sudo yum install python3} @) 
\end{code} 
% $

\subsection{Python3 on Macintosh}
On MacOS, install matplotlib for Python 3.3 with:
\begin{code}
$ (@ \hl{sudo port install py33-matplotlib} @)
\end{code}
%$

\section{Setting Up your Computer}
This section provides recommendations for setting up your computer to
make the most of the examples and exercises in this book. Remember,
these are \emph{recommendations,} not requirements.

\subsection{Windows}
Once you have installed Python, you may wish to modify your Windows
console so that it has 132 rows of text and 9999 lines of
scrollback. You can do this by running the |cmd.exe| program from the
Start menu then right-clicking on the window's titlebar and selecting
the ``Properties'' menu, as shown in \figref{ch-1/windows-console}.

Windows users may also wish to become familiar with the Windows
PowerShell, a replacement for |cmd.exe|.

\sgraphic{ch-1/windows-console}{Setting your Windows console for 132
  rows of text and 9999 lines of scrollback will improve the usability
  of many text-based commands.}

If you do not have a C++ compiler, you should install one. Many
Windows users will feel most comfortable with an integrated
development environment (IDE). We
recommend:

\begin{itemize}
\item Code::Blocks\furl{http://sourceforge.net/projects/codeblocks/},
  a popular Open Source IDE that includes the mingw compiler suite 
\item Orwell Dev-C++
  (\url{http://sourceforge.net/projects/orwelldevcpp/})
\item Visual Studio Express 2012 for Windows
  Desktop\furl{http://www.microsoft.com/visualstudio/eng/products/visual-studio-express-products},
  Microsoft's compiler, allows development in C++, C\# and
  VB.NET. Runs on Windows 7 and later versions.
\end{itemize}

\subsection{Linux}

\subsection{MacOS}

For Mac systems we recommend installing Python 3 with the MacPorts
system. Download MacPorts from \url{http://macports.org/} and then type:

\begin{code}
$ (@ \hl{sudo port install python33} @) 
\end{code} 
% $


