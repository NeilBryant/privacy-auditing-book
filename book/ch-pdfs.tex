\chapter{Private Data in PDFs}

In April 2011, the UK Parliament posted an
Adobe Acrobat Portable Document Format (PDF) file on its public website detailing key vulnerabilities of
UK nuclear subs. The electronic document had been cleared by the
British Ministry of Defense, where an official had been charged with
removing the sensitive portions, but the redactions were made incorrectly: the sensitive text had
simply been obscured with a black box, and could be
recovered by  copying the text out of the PDF file and pasting
it into a word processor. The
result was a document that appeared to  be properly redacted, but
which still contained state secrets.  A follow-up investigation by
\emph{The Daily Telegraph} found similar  PDF files on other UK government websites
found four other documents where sensitive
information was improperly ``redacted''\cite{telegraph-april2011-secrets}.

Leaks of sensitive information  in PDF files are frequently
newsworthy, but they are hardly new. In October 2003, the US Justice
Department posted a report on its website about the Department's
internal efforts at increasing racial diversity---but every conclusion and
recommendation was blacked out. The
redaction was imperfect: journalist
Russ Kick discovered that the black boxes
could be easily removed~\cite{nyt-diversity-critical}. 

``PDF'' is the Portable Document Format, developed by Adobe in the
1990s as a universal file format for describing printed documents. 

\section{Improperly redacted data}
\section{High-resolution JPEGs}
\section{High-resolution line drawing}
medical information leakage.
\section{Metadata}
\section{References}

\bibentry{nsa-pdfs}


