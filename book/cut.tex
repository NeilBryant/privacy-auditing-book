% This file is for information cut from a chapter that might be good
% if moved elsewhere...



% Cut from Chapter 1

\subsection{Personally Identifiable Information?}
Most of the leaks described in the previous section share a common
theme: each involved the collection and possible release of personally
identifiable information (PII).

\tk{Explain what PII is}

\subsubsection{Legal definitions}

\tk{Explain legal definition of PII}

\subsubsection{Practical definitions}

\tk{Provide a practical definition of PII}

\subsection{Practical Concerns}


\subsubsection{Reuse of identifiers - at the same time, at different times}
\subsubsection{Shared IDs}
\subsubsection{specificity of identifiers; birthday problem}
\subsubsection{IP addresses}
\subsubsection{Identifiers: IPv4, IPv6, Mobile ID, SIM cards, Phone \#s, email addressses}
\subsection{GUIDs and identifiers}
\subsection{What is ``Unique''}


Other examples of privacy leaks include:
\begin{itemize}
\item Microsoft leaking windows registration information (cite)
\item VA Laptop
\item In May 2013 a firm called Decipher Forensics revealed that Snapchat, an application that sends photos with a
  self-destruct timer, didn't actually delete photos as
  promised. Instead the program simply renamed the photos so that they
  could not be seen with normal tools on the telephone\cite{ksl-snap-chat}.
\end{itemize}



\section{Privacy and Public Policy }

\subsection{Non-Technical Privacy Audits}

\subsection{Allowable Leaks}

\section{Unnoticed data leakage - why it's a problem}
\subsection{Example: Geolocation Data in JPEGs}
  - Show how to find where a photo was taken with Preview on a Mac. 
\subsection{Example: Blacked out text in PDFs}
EEOC

Senate committee

