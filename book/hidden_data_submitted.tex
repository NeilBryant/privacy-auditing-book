
But it's not just Government bureaucracies---\emph{The Times}  made the
same mistake when it published what it claimed was a US intelligence
report from the 1950s on the paper's
website in 2000~\cite{nyt-unediting-2003}. \emph{The Washington Post}
  made a similar mistake in 2002 when it posted a scan of the DC
  Sniper's demand letter\cite{internet-forensics}.

Adobe introduced the PDF format in 1993. Since then, PDF has become an
international standard for distributing born-digital documents, scans
of paper documents, and even electronic
forms~\cite{ISO32000-1:2008}. Ironically, PDF gained significant
popularity precisely because it was a way to distribute documents
without the risks associated with the Microsoft Office file
format. But as these examples show, PDF is a complex format that can
contain information that is invisible but nevertheless easily
extracted.

This article explains why PDF files have privacy risks that continue
to befuddle users to this day. It also explores the metadata risks
caused by the Office file format. Finally, it presents some of the
technical solutions that vendors have developed to address these
problems, an suggests a better solution that word processors could
adopt today to solve the problem.

PDF files are not \emph{inherently dangerous}, perhaps, but they
demand respect. Security and privacy professionals need to
understand why such leaks keep happening and how to stop them.

\section{Privacy Leaks in PDF}

Adobe created PDF in the 1990s as a single distributable electronic
container to allow easy viewing and printing of electronic documents
on any modern computer system without the need to install software or
fonts other than a general purpose PDF reader. Adobe released programs
for making and reading PDFs, originally called Acrobat Distiller and
Acrobat Reader. It also published the PDF specification and
licensed the patents required to implement it. Today there are
numerous programs that can create, display and even edit PDF files,
and the format has been adopted as an international standard (ISO
32000-1:2008).

Before PDF, electronic documents were typically distributed either as
plain text or as PostScript files. \emph{PostScript} is a programming
language that Adobe Systems created in the 1980s. The key idea was
that the language was interpreted on the printer, rather
than by the computer, allowing relatively small amounts of data to be
sent to the printer, which would then render the page at the highest
resolution the device supported.

Apple licensed PostScript for its LaserWriter, the first mass-market
laser printer. There were inkjet printers, plotters, and large-scale
laser printers before the LaserWriter, but each had their own
language, and many were quite limited. With PostScript, application
programs didn't have to separately understand how to generate print
codes for each device or how to transform complex graphics into a
rasterized bitmap. Instead, an application program could simply
generate PostScript and know that the resulting page would be
correctly rendered. One of PostScript's most powerful features turned
out to be the ability to distribute high-quality line art as
encapsulated PostScript files that could embedded in other documents,
scaled, transformed, and then printed---all with little work on the
part of the application program. This, combined with the language's
support for high quality typography, made PostScript one of the key
enabling technologies of the 1980's desktop publishing boom.

But PostScript's power made the language
less than ideal as a universal document format. First, the language
 is  verbose, so PostScript files can still be quite
large. Second, the PostScript execution context is not reset
after each page, so document pages can not be displayed out of
order: showing the contents of page 100 required first processing pages 1 through 99. PostScript also
lacks support for many features demanded of a modern electronic document format,
such as encryption, digital rights management, and electronic forms.

PDF is based on PostScript, but it is a specialized file format. Like PostScript, PDF has
commands for drawing lines, embedding fonts and displaying text. But PDF also has direct support for
document structure; a variety of image and video compression
algorithms; electronic forms; digital rights management; and
cryptography. PDF allows embedded JavaScript for form validation
and other kinds of automation. Any page in a PDF document can be
rendered without reference to any other. 

PDF's rich support requires a complex reader, and such readers inevitably have flaws. Many of these flaws
have been exploited by attackers using cleverly constructed
documents~\cite{5705599}. But whereas PDF exploits depend upon
implementation errors, PDF privacy leaks typically result from
properly formed files  containing information that is invisible but
readily extracted. 

\subsection{Creating PDFs}

Users typically create PDF files in one of two ways. Users can
generate a PDF directly from an application program like a word
processor or web browser.  For example, today numerous drivers allow
users to ``print'' directly to a PDF file, and some application
programs can export directly to the format. PDF files created in this
manner contain the actual fonts and characters that are rendered by the
output device and have sharp, crisp letters, text that can be
selected and copied, and are searchable. These PDF files are 
considered \emph{accessible}, since people who are vision impaired can readily use
the documents by enlarging the text or using screen readers.

A second way to create PDF files is to scan a paper
document, producing a PDF file that is essentially a collection of
photographs. Even when scanned at 300 dots per inch, the
resulting document can appear pixelated, mushy, and difficult to read on a
computer screen, since the PDF reader is displaying a photograph of text,
rather than rendering the text. Such
text is not accessible---it cannot be selected, copied, searched, or
read with a screen reader. Such files also tend to be 10 to 100 times
larger than the PDF files created with the first
method. Unfortunately, many PDF files in circulation are scans of
paper documents because
the original paper documents were not prepared using a computer,
the original electronic files are no longer  available, or the
individuals who prepared the files did not have the tools or 
training to create a PDF file directly.

Adobe's Acrobat Pro and other PDF processing programs have the ability
to ``Recognize Text'' in scanned PDFs with an optical character
recognition (OCR) engine. Acrobat Pro can then modify the PDF file and
place the recognized text in a second layer underneath the scanned
image. The result is a scanned PDF that is searchable but that
preserves the original image. The recognized text can even be copied
into another program. This is an example of hidden text with a
genuinely useful purpose!

% With Acrobat 9 introduced a variant of OCR called
% ``ClearScan'' that creates a custom font for the each scanned
% document based on the character bitmaps present in the document. The
% program then replaces matching bitmaps on each page with characters
% from the new font.  The result looks better, but when there is an OCR
% error ClearScan can inadvertently produce apparent typos or even semantic changes in the resulting
% document.

Surprisingly, the PDF standard itself does not make a distinction
between PDF files that are accessible and those that aren't. From the
point of view of the standard, a PDF is simply a file that contains
\emph{objects} (e.g.\ text drawing commands and fonts); a \emph{file
  structure} that describes how and where the objects are stored in
the file; a \emph{document structure} that describes how the objects
are used to create pages; and one or more \emph{content streams} that
describe how to display pages on the screen and how to print them.  The
power of  PDF comes from the ability to compose  these features in  many different ways. 


\subsection{Textual Privacy Leaks}


The behavior of Word is particularly pernicious because the resulting
text looks very similar to a document that has been redacted through
the official procedures used by many governments---obscuring sensitive
text with black boxes. Furthermore, there would likely be no
significant privacy leak if the document were printed and
distributed as hard copy---or if it were printed, scanned, and a non-accessible PDF
distributed. It is only the distribution of the PDF created directly
from the word processor with black boxes over black text that results in the leak.

The easiest way to recover the hidden text is with Acrobat Reader's
copy-and-paste commands: copy the blacked out text and paste into
another document. A More dramatic demonstration is to open the PDF
with a PDF editing tool such as Adobe Illustrator or Inkscape. These
tools understand the internal structure of PDF files and allow objects
on each page to be individually manipulated. With Inkscape, for
example, the box can be selected and then changed from black to
fuchsia (Figure~\ref{fuschia}), resulting in the highlighting of text
that was intended to be redacted.

\begin{figure}
\includegraphics[width=\textwidth]{demo/fig1-before.png}\\
\includegraphics[width=\textwidth]{demo/fig1-after.png}
\caption{Black boxes created with the Text Highlight tool can be
  readily changed to a different color with PDF editing tools such as Inkscape.}\label{fuschia}
\end{figure}

The Text Highlight tool is not the only source of textual privacy
leaks. In the case of the redaction errors at \emph{The New York
  Times} and \emph{The Washington Post}, both papers apparently
scanned a paper document, then
drew black boxes on the scanned pages using Adobe Photoshop.  Instead of drawing the black boxes directly on the
scanned bitmap, however, the boxes were placed in another Photoshop
layer---the application program's default behavior, so that
the files can be re-opened and the individual layers re-edited. Of course, this is not what
the editors intended. The problem could have
been averted if the Photoshop operator had simply ``flattened'' the
layers in the PDF file before it was made available on the newspaper's web server\cite[p.152]{internet-forensics}.

% \emph{The Washington Post} made a similar mistake on October 26, 2002,
% when it published the scan of a four-page demand letter from the DC
% Sniper on the paper's website. The sniper's letter included a demand
% for \$10 million that the murderer wanted to be deposited in a debit
% card account. \emph{The Post} attempted to redact
% the financial details with black boxes on the PDF file, but once
% again, the PDF file was not flattened and the boxes were easily removed~.

\subsection{Image Privacy Leaks}

Another way that private information can leak in a PDF is through the
embedding of digital photographs that have very high resolution images or
sensitive metadata.

For example, when a Macintosh user drags a JPEG digital photograph
from the desktop into a Word Mac 2011 window and saves the file, Word
embeds an exact copy of the file inside the resulting Office Open XML
(.docx) file. When Word produces a PDF file from the document, it
embeds the original JPEG directly in the PDF as a PDF object. Word
then places instructions in the PDF to scale, rotate and crop the JPEG
as the user had directed. The original JPEG can be extracted using
Acrobat Pro or open source tools such as \texttt{pdfimages}. Of
course, the image can also be extracted directly from the (.docx)
file, should that file be mistakingly distributed.

Embedding JPEGs directly in PDFs
can be both annoying and  risky to
users. The annoyance is that the resulting document files  can
be much larger needed: a typical high-resolution JPEG
can be 3000 pixels across and 3
megabytes in size or larger. If the image is shrunk  to two
inches, the result is a \emph{super-resolution} image with 1500 dots
per inch of information (only 150 dots per inch are required for
photographic quality production). As a result,  organizations exchange
document files  much larger than necessary because the files
contain embedded super-resolution photographs. 
% This can be a
% problem especially for mobile users, but it also increases the costs of backup,
% data storage, and and other operations for all users.

The privacy risk is that super-resolution images can reveal information
without the knowledge of the document's author, editor or
publisher. People in a photo that are too small to identify might be
clearly identifiable if the photo is enlarged. High-resolution photographs of a
person's face can sometimes be used as a
biometric. License plates, computer screens, and sometimes even papers
on a person's desk can sometimes be enlarged and read. 

Embedding unprocessed JPEGs also frequently results in the
embedding the JPEG's Exchangeable image file format (Exif) metadata. Exif
information can include the model and serial number of the camera that
took the picture, the date and time that it was taken, and more. Modern digital cameras, including those on most
cell phones, can also embed GPS coordinates.

The treatment of embedded JPEGs by desktop applications is
inconsistent. Both Word and Apple Pages embed the
unmodified JPEG in both document files and in PDFs generated from the ``print''
menu. But when the Pages ``export PDF'' command is invoked and the export
quality is changed from ``Best'' to ``Good,'' the embedded
JPEGs are converted to a lower resolution and the metadata is stripped.
Both Microsoft PowerPoint and Apple Keynote reduce the resolution of
high-resolution JPEGs embedded in slide presentations and exported as
PDFs. Overall,  it appears that the metadata stripping is an
inadvertent side-effect of the process by which the application
creates a lower-resolution JPEG, rather than an intentional effort on
the part of the application authors to strip privacy-sensitive
information. 

\subsection{Cropping Privacy Leaks}

Cropping and masking tools can also result in privacy leaks. This can
happen when the original privacy-sensitive object is placed directly
into a document file (and in a resulting PDF) and the cropping or
masking transformation happens when the object is displayed. As is the
case with text that is hidden-but-present, PDF manipulation tools can
readily undo the transformations and reveal the original content.

Programs are inconsistent in acknowledging the risk. When a user
instructs Apple Preview to crop a PDF, Preview
displays an alert box warning ``the content outside the selection is
hidden in Preview, but you might be able to view it in other
applications.'' Adobe Acrobat Pro provides no such warning.

% \begin{figure}
% \includegraphics[width=3in]{demo/apple-alert}
% \caption{Apple's Preview program displays an alert box when the user
%  attempts to crop an image informing the image that content remains
%  in the PDF file. Adobe Acrobat Pro also leaves content in the file
%  when a PDF is cropped, but it does not warn the user}\label{crop-warning}
% \end{figure}


\subsection{A PDF Privacy Experiment}
One can readily test word processing software to see if it will embed
black-on-black text into a PDF. To do so, create such a file in a word
processor, produce a PDF, and attempt to find the hidden text in the
resulting file. Such tests are important, because software changes on
a regular basis---far too fast for academic literature to keep up. As
a result, privacy conscious users must test their tools to determine
when the tools leak private information.

For this article, six test documents were created using Microsoft Word
(Mac and Windows), Google Docs, Apple Pages , LibreOffice, and \LaTeX.
Each document had a single line of text consisting of a number, the
name of the product, the word ``VIS$\blacksquare$BLE'' with a yellow
highlight, the word ``HID$\blacksquare$DEN'' with a black
highlight, and the word ``test.'' (In each case, the\ 
$\blacksquare$\ is replaced with the number at the beginning of each line). Thus, each
file has two kinds of highlighted text---one that a na\"ive user would expect to find in
a file, one that such a user would not---and each instance of highlighted
text is different, allowing the provenance of each piece of hidden
text to be tracked if  all are combined into a single file. 

All of the programs tested were found to embed highlighted text inside the
PDF file, even when the text had the same color as the
background (Figure~\ref{demo}). This is the cause of many privacy
leaks. (Readers are invited to scan the PDF of this article to see
if the text is still present.) This poses an important
question: what \emph{should} be the correct behavior?  We will return
to this question later.

\section{Privacy Leaks in Document Metadata}

``Metadata'' is a term that has come to mean data that describe other
data such as a title describing a document, a phone number describing
a recorded audio call, or a date describing a table of scientific
measurements. Document metadata can be stored within a document itself
or separate from the document. For example, file systems store
\emph{extrinsic metadata}~\cite{garfinkel:ascription} such as the
creation date and modification date of each file. Some document file
formats also contain internal or \emph{intrinsic} metadata.

Both PDF and Microsoft Office have support for explicit metadata
include the document title, author, producer, creator, creation date,
modification date, and keywords, as well as user-defined
metadata. Additional human readable metadata that cannot be readily
retrieved includes copyright strings, color space names, file names,
and the operating system on which the file was created.

Many metadata privacy leaks are the result of documents being made
available in the Microsoft Office binary file format. Of these, the
most significant leaks have been the result of embedded change
tracking information, although other metadata has caused problems for
Microsoft users as well.


\subsection{Comments Change Tracking Leaks}

Microsoft Word introduced the ability to track changes with Word 6 on
the Macintosh and Word 95 on
Windows\footnote{\url{http://support.microsoft.com/kb/291337}
  and http://support.microsoft.com/kb/189041}. Today change
tracking is widely used in business and government. Change
tracking must be specially enabled on a per-document basis. Once
enabled, Word records every addition, deletion, and formatting
change. For each change, Word records the username of the account that
made the change and the time the change was made. Word also allows
users to insert non-printing comments into the document.

With Word 2002, Microsoft gave Word the ability to have ``hidden
tracked
changes.''\footnote{\url{http://support.microsoft.com/kb/291337}} In
modern versions of Word this is implemented with a pop-up menu that
allows the user to change the display mode from ``Final Showing
Markup'' to ``Final.'' Doing so causes tracked changes and comments to
disappear from the screen and printed copy. The changes are still
present in the Word file, however, and can be easily displayed if a
recipient of the document  changes the view mode back to ``Final.''

One of the most significant cases involving hidden change tracking
information in a Microsoft Word file involved the pharmaceutical maker Merck and
an article about the drug rofecoxib (Vioxx) that was submitted to
\emph{The New England Journal of Medicine} on May 18, 2000. The
article reported on the result of a study involving 8076 patients that
compared the effectiveness of rofecoxib with naproxen, an
over-the-counter remedy, and found that rofecoxib resulted in
significantly fewer ``clinically important upper gastrointestinal
events'' such as stomach ulcers than
naproxen~\cite{bombardier-2000}. Rofecoxib had been approved by the US
Food and Drug Administration for just a year; within five years, more
than 2 million people were using the drug. 

On September 30, 2004, Merck withdrew rofecoxib from the market,
citing a study published the day before showing that people
taking rofecoxib had twice the incidence of heart attack or stroke as
those taking a sugar bill~\cite{npr-vioxx}. Millions of patients had
been put at risk.

After Merck pulled rofecoxib from the market, the executive editor
of\emph{The New England Journal} reviewed the file from the 2000
article and discovered a floppy disk containing the submitted version
of the manuscript. According to change tracking information in the
document, a table detailing cardiovascular events had been deleted
just two days before the manuscript had been submitted to the
journal~\cite{nejm-concern}. According to the journal's editor, a
table detailing problems with rofecoxib---the same problems reported
in 2005---were deleted by a user named
``Merck''~\cite{forbes-mercks-deleted-data}. 

If that information had
been published in 2000, the drug might never have become so popular.
Merck ultimately paid \$4.85 billion in 2007 to settle 27,000 lawsuits
by those claiming to be injured by rofecoxib and \$950 million more to
settle criminal charges resulting from its
actions~\cite{nyt-vioxx-settlement}. One of the key pieces of evidence
was the original Microsoft Word file.

% In a less spectacular case, the SCO Group, a company that acquired
% rights to a commercial version of the Unix operating system previously
% sold by the Santa Cruz Operation, filed suit against DaimlerChrysler
% in March 2004 for alleged copyright infringement resulting from the
% use of the Linux operating system. The SCO Group provided a copy of
% the lawsuit to journalists, who promptly discovered that hidden
% tracked changes were present. The tracking data revealed that the SCO
% Group had spent considerable time focusing on Bank of America as a
% possible defendant, and had only changed the name to DaimlerChrysler a
% few weeks before the suit was filed. The file also contained internal
% comments passed between authors working on the
% lawsuit~\cite{sco-hidden-text}.
% 

\subsection{Other Office Metadata}

Microsoft Word files contain significant metadata other than change
tracking information, such as the document's author, keywords,
contents, the last time the file was printed, and path information for
previous saves. This information can reveal information that the
author wishes to remain confidential.

For example, on January 30, 2003, UK Prime Minister Tony Blair's
office released a Microsoft Word document entitled ``Iraq --- Its
Infrastructure of Concealment, Deception and Intimidation,'' claiming
it was a previously classified document detailing the case for
invading Iraq based on the country's likely possession of weapons of
mass destruction.

Days after the document's release, Glen Rangwala at
Trinity College determined that the document was largely
plagiarized from documents that were freely available on the
Internet~\cite{rangwala-2003}.  Rangwala's analysis, based solely on
the content of the document, caused considerable embarrassment for the
UK government.

Four months later, computer security expert Richard Smith
announced that the actual Microsoft Word file Blair's Office had
released contained hidden information as well---metadata tracing the
last 10 revision histories of the document. 

Word's ``revision histories'' feature is different from change
tracking---it does not track the actual text, but only the date of the
save, the file name, and the username of the person who performed the
save. Based on the times and file names that had been used, it was
possible to show when the document had been copied to a floppy disk,
reportedly so that US Secretary of State Colin Powell could have a
copy for a presentation at the United Nations\cite{smith-10}.

A recent study of 15 million Microsoft Office files available freely over the
Internet found that 97\% included significant metadata, and 19\%
included all 10 revision histories~\cite{6503202}. The authors found
that they could readily infer collaborators between corporations and
the US military, and cross-correlate between document authors with
Twitter accounts. ``Our study raises major concerns about the risks
involved in privacy leakage, due to metadata embedded in documents
that are stored on public web servers,'' the authors concluded.


\section{Solutions}
Broadly speaking, there are two approaches for addressing user
interactions with computers that have potential security problems:
train the users so that they do not perform those actions, or design
systems so that performing those actions is unlikely or
impossible. 


\subsection{Training}

Until now, the computer industry has largely focused on
the use of specialty software and training to solve the problem of
data leakage in complex documents. 

For example, Adobe Acrobat versions 8 and above includes a feature for
redacting data from PDF files. Acrobat redaction is a three step process
that involves first marking the pages for redacting, applying the
redaction, and then optionally removing sensitive information from the
PDF file. The resulting file has black boxes in place of the
removed text, but unlike boxes applied in a word processor or
PhotoShop, there is no text underneath!  Acrobat XI further allows
the user to annotate the black boxes with exemption codes, as may be
necessary when redacting documents that are being released as a result
of a US Freedom of Information Act request or when preparing documents
that are partially protected by the US Privacy Act. 

% \begin{figure}
% \includegraphics[width=4in]{demo/acrobat-redaction}
% \caption{Acrobat XI's Redaction Tool Properties
%   panel}\label{acrobat-redaction}
% \end{figure}

Microsoft's ``Crabby Office Lady'' advice columnist Annik Stahl
advised users that they should remove
tracked changes, comments, and hidden text before sending out
documents such as resumes, annual reviews, and contract
bids~\cite{microsoft-track-changes}.


\subsection{Software Modification}

An alternative to training is to modify
word processors to make these kinds of redaction errors less likely or
even impossible.

For example, when a user with LibreOffice Writer  attempts to change the
background of black text from white to black, the program
simultaneously changes the color of the text from black to
white. This is implemented not as a special case, but through the use
of a default text color called ``Automatic,'' which changes the text
color ``so the text is always distinguishable from the background
color''~\cite{libreoffice-accessibility}. The feature is designed to
promote accessibility, but has the side effect of requiring two steps
to create black-on-black text---one to set the background, and one to
change the text color. The developers could go further, and simply
disallow text and background to be set to the same color as a safety
measure.

Another solution, which could be implemented by every word processor,
would be for the programs to simply detect when a letter is being printed on the
same-color background and to simply suppress the printing of the letter.

LibreOffice helps address the problem of super-resolution images in
PDF files by bring the issue to the attention of users. 
LibreOffice's PDF export dialogue allows the user to set both the JPEG
compression quality and to specify a final image resolution in dots
per inch. This is superior to
Apple's PDF export dialogue, which only allows the user to specify
an ``Image Quality'' of ``Good,'' ``Better'' or ``Best,'' which provides no way for a user to readily
infer that image quality has something to do with image resolution or,
by extension, privacy.

Experiments with LibreOffice's PDF export function revealed that the
program strips Exif information if it modifies the embedded JPEG but
not otherwise. As before, the program's privacy functionality appears to
be a side effect of features designed to files from being
unnecessarily large, rather than an explicit attempt to limit the
unnecessary spread of potentially sensitive information. 

The current versions of Microsoft Word on both Windows and Macintosh
have ``privacy options''for addressing sensitive metadata. On
option, ``remove personal information from this file on save,''
removes the names of the computer's user from document
metadata. Tracked changes are not removed, but the name is changed
to ``Author'' and the time of the modification is removed
(Figure~\ref{msword-remove-demo}). On Mac Word 2011 the option is found by
select Word/Preferences/Security from the main menu or Options/All
Options/Security from the ``Save As'' menu. On Word 2010 for Windows
the option is found by selecting the File/Options/Trust Center/Trust
Center Settings/Privacy Options. A special ``Document Inspector''
claims to check for and remove 10 different kinds of hidden
metadata. 

Disturbingly, the Microsoft Word user interface gives the impression
that the program will also detect and eliminate text that is hidden or
invisible, such as is the case with black-on-black text that is
printed to a PDF. For example, interface offers that it 
``Inspects the documents for objects that are not visible because they
have been formatted as invisible. This does not include objects that
are covered by other objects.'' The hidden text inspector claims that
it ``Inspects the documents for text that have been formatted as
hidden.''  Tests show that these features do
not discover text on same-color background, as word only identifies text as ``hidden'' when the
text font properties are changed to ``hidden,'' a special Word font
property that prevents the text from printing. The apparent confusion results
from Microsoft's use of the word ``hidden'' as a proper noun to
describe a feature of Microsoft Word, rather than as an adjective to
describe a property of text that is present in a file.

\begin{figure}
\includegraphics[width=2in]{demo/msword-change-before}\\
\includegraphics[width=2in]{demo/msword-change-after}\\
\caption{Microsoft Word's ``remove personal information from this file
  on save'' option removes the name of the person who changed the
  document, but leaves change tracking information.}\label{msword-remove-demo}
\end{figure}

As a result of the metadata privacy incidents discussed in this
article and other situations, several software vendors now offer
enterprise metadata removal tools. For example, the PayneGroup's
Metadata Assistant can be configured to remove metadata from word and
PDF files on demand or automatically, such as when files are sent by
email to another organization. Metadata removal can be performed on
the end-users computer or on a server. According to the American Bar
Association, 17 states now hold that attorneys have an ethical
requirement of exercising ``reasonable care'' in removing metadata
prior to transmitting it\cite{aba-survey}. Of those, six hold that
recipients of documents with metadata may mine, nine hold that such
practice is ethically prohibited, and two hold that the situation is
case-specific. A 2010 survey by the American Bar Association found
that 59\% of the respondents' firms had some kind of specialized
metadata removal software available, up from 46\% the year before.


% \begin{figure}
% \includegraphics[width=4in]{demo/msword-document-inspector1}
%\includegraphics{demo/msword-document-inspector2}
% \caption{Microsoft Word's ``Document Inspector'' claims that it will
%  identify invisible content and hidden content. However, the feature
%   will not identify white text on white background or black
%  text on black background as being invisible or hidden.''}\label{msword-bug}
% \end{figure}

\section{Conclusion}

Both PDF and Microsoft Office files have the ability to contain
information that is essentially invisible to the individual preparing
the electronic document yet easily extracted by a recipient. Today a
significant number of documents are available on the Internet for
download that contain revealing metadata, and there continues to be
occasional but high-profile cases involving the improper redaction of
PDF documents resulting from the improper use of the Text Highlight
tool to hide information. To address this problem, vendors and users
have largely relied on increased training and specialty tools. 

A better approach would be to modify tools so that the underlying data
model is in line with what is presented to users through the user
interface---that is, by making it harder for users to produce
documents with hidden information. In the absence of such redesign,
embarrassing data leakage incidents are sure to continue.

% http://www.theregister.co.uk/2004/05/13/student_unlocks_military_secrets/

% \subsection{Related Work}
% Stevens discusses a variety of such
% attacks, including vulnerabilities in data compression algorithms and
% PDF's JavaScript engine\cite{5705599}
% 
% In some specific cases even relatively low-resolution figures may
% cause significant privacy leaks. In 2006 Brownstein, Cassa and Mandi
% reported in \emph{The New England Journal of Medicine} that they could
% readily convert the dots from a map showing the location of infected
% patients to their street addresses using standard geolocation
% techniques, provided that the map was published at sufficiently high
% resolution\cite{brownstein-2006a}. In an experiment, the researchers
% were able to identify 79\% of 550 simulated patients shown on a map of
% Boston, provided that the map was printed at 266 dots per inch (the
% minimum resolution required by the \emph{Journal}). The resulting JPEG
% was only 712kb. The researchers identifying 19 articles published
% between 1994 and 2005 containing such maps, and a total of 19,000
% addresses that could be inferred from them. In a follow-up article
% with an improved geolocation technique, the researchers reported that
% they could precisely identify patent addresses 79\% of the time with
% publication quality maps, and 26\% of the time with ``presentation
% quality'' map, such as might be found in a PowerPoint
% presentation\cite{brownstein-2006b}.


% Cranor Engineering Privacy \cite{4657365}

% Steg in OLE files Erbacher \cite{5341560}

% http://office.microsoft.com/en-us/help/find-and-remove-metadata-hidden-information-in-your-legal-documents-HA001077646.aspx
% http://www.firstamendmentcenter.org/ariz-high-court-hidden-data-in-public-records-must-be-disclosed

\begin{figure}
\includegraphics[width=\textwidth]{demo/1-wordmac2011.pdf}\\
\includegraphics[width=\textwidth]{demo/2-word2010.pdf}\\
\includegraphics[width=\textwidth]{demo/3-googledocs2013.pdf}\\
\includegraphics[width=\textwidth]{demo/4-applepages.pdf}\\
\includegraphics[width=\textwidth]{demo/5-libreoffice.pdf}\\
\includegraphics[width=\textwidth]{demo/6-latex.pdf}
\caption{Demonstrations of using the highlight tool to create redacted
  text. The misalignment of the Google Docs highlight appears to be a
  bug in Google's software. The reader is invited to attempt to recover the text under the
black boxes.}\label{demo}
\end{figure}

% \begin{figure}
% \includegraphics[width=\textwidth]{demo/5-libreoffice-fixed.pdf}
% \caption{LibreOffice's ``automatic'' default text color changes text from
%   black to white when the background color is changed from white to
%   black. }\label{fixed}
% \end{figure}

% \begin{figure}
% \includegraphics[width=\textwidth]{demo/4-libreoffice-screen}
% \caption{LibreOffice's PDF export page allows the user to specify both
%  the JPEG compression quality and the final image resolution,
%  avoiding the problem of super-resolution images being embedded in
%  regular PDF documents.}\label{libreoffice-export}
% \end{figure}


\bibliographystyle{IEEEtran}
\bibliography{../../bib/garfinkel,../../bib/dfrws,hidden_data}

\shadowbox{\begin{minipage}{\textwidth}
URLs for documents mentioned in this article
\\
\begin{description}
\item[\url{http://www.user-agent.org/washpost_sniperletter.pdf}]~\\DC
  Sniper letter.
\item[\url{http://cryptome.org/iran-cia/cia-iran-pdf.htm}]~\\Copies of
  New York Times PDF files purporting to describe US involvement in Iran, 1952.
\item[\url{http://www.computerbytesman.com/privacy/blair.htm}]~\\Description
  of metadata found in 10 Downing Street dossier on Iraq's security
  and intelligence organizations, including a copy of the original file.
\item[\url{http://digitalcorpora.org/corp/nps/files/2008-pdfs/diversityanalysis-orig.pdf}]~\\
 Redacted May 2002 KPMG Consulting report on Diversity within the US
  Department of Justice Attorney Workforce.
\item[\url{http://www.justice.gov/dag/readingroom/diversityanalysis.pdf}]~\\
Unredacted KPMG report.
\end{description}
\end{minipage}
}



\end{document}



TSA to Conduct Full Review After Leak of Sensitive Information
http://www.usnews.com/news/articles/2009/12/07/tsa-to-conduct-full-review-after-leak-of-sensitive-information
- Improperly redacted document stamped ``Sensitive Security
Information'' was posted on a Federal Business Opportunities website.

% http://news.bbc.co.uk/2/hi/europe/4506517.stm
Readers 'declassify' US document - 2 May 2005
Pentagon PDF document.

% http://news.cnet.com/AT38T-leaks-sensitive-info-in-NSA-suit/2100-1028_3-6077353.html
% At&T - Govenrment wiretapping program. May 26, 2006

% http://en.wikipedia.org/wiki/Sanitization_(classified_information)

% http://www.theregister.co.uk/2011/04/18/dnsr_report_declassified_not_redacted/
% 2009 UK ``Restricted'' report was released. as PDF
% http://www.telegraph.co.uk/news/uknews/defence/8457506/Secrets-put-on-internet-in-Whitehall-blunders.html
% - Daily telegraph discovered numerous lapses
% http://www.dailystar.co.uk/news/view/186568/Nuclear-sub-secrets-revealed-by-MoD-schoolboy-error-/
% http://nakedsecurity.sophos.com/2011/04/18/how-not-to-redact-a-pdf-nuclear-submarine-secrets-spilled/


%%  LocalWords:  extractable searchable redactions distributable
