\chapter{Designing a good forensics experiment}

This chapter explores what is involved in creating a good computer
forensics experiment.

\section{Why experiment?}

why should you experiment?

\section{What is the purpose of the experiment? - what you can provide and what you can't}
\section{Start with a wipe}
\section{Use Self-Identifying Data}

What self-identifing data is.


% M-sequences in radar
% KW37 - keying is for a day; you can join anytime.
% Watermarking - covert and robust go together
% hackmem search algorithm

\section{Sampling vs. Complete Analysis}

sometimes you can analyze all the data, but frequently you can't

\section{Working with large amounts of data}
 - What's large?
 - 4GiB limitations (FAT32, ZIP vs. ZIP64)
\section{Error Rates}
\subsection{What is an error rate}
\subsection{Error rates from hardware}
\subsection{Error rates from sampling}

