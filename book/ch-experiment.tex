\chapter{Designing experiments for privacy auditing}

In everyday speech people frequently use the word ``experiment'' to
mean that they are going to ``try something new.'' For example, a
person might conduct an ``experiment'' in the kitchen by adding more
salt and pepper to the fish---more salt might be a good thing, but too
much salt might make the dish inedible.  Experiments such as these are
hard to carry out systematically, though, because two many factors are
changed at once. The flavor of the dish might be subtly influenced by
the specific fish---where it lived or its age, for example. The flavor
might also depend on the what the taster ate earlier in the
day. Experimental results can also be influenced by expectation---if
we know that we put more salt in the dish, we might expect it to taste
saltier and judge the dish accordingly. So while experimenting in the
kitchen by adding a bit more salt to the fish might be informative,
considerably more effort is required if we want to discover something
that can be generalized to other circumstances. To do that we need to
use the scientific method.

Scientific experiments involve more than simply trying something new
to see what happens. Whereas in the kitchen we might start out with
the question ``what can I do with the ingredients that I've got?'',
scientific experiments are designed to learn or confirm a specific
facts. To do this experiments typically combine some kind of
\emph{test} or \emph{intervention} with one or more
\emph{measurements} or \emph{observations}. Experiments should be
repeated to account for minor variations or errors, and they should be
described in sufficient detail so that they can be repeated by
others.

\section{The Experimental Steps: A Worked Example}

Experimentation is one of the core technique that we use in technical privacy
auditing to understand how systems retain or transmit personal
information. In most cases these experiments will be quite
straightforward: 

\begin{itemize}
\item We will use a digital system.
\item We will analyze the system with digital forensics tools to see
  what kind of personal information is retained or transmitted.
\end{itemize}

Things are rarely so simple, of course. Digital computers are
complex systems. Finding information on a system doesn't mean that we
put it there---the information might have been there from the
beginning. Likewise, \emph{not finding} information is no guarantee
that the information is not present. The information may be present
but the tool is unable to extract it; the operator may use the tool
improperly; or the tool may extract the information but in a form that
the operator does not recognize.

In practice, technical privacy auditing requires more than running a single
experiment. We'll need multiple experiments aimed at uncovering a wide
range of digital phenomena. We'll need to test the digital system
to see how it operates normally. We'll need to test our tools to see
if they can recover the specific kind information for which we are
looking.

\subsection{Private Browsing Mode}

To make this example more concrete, let's consider a researcher who
wants to test the ``private browser mode'' on web browser to see if
the browser retains information that it should not.

The concept of a privacy mode for web browsing was introduced by the
Apple's Safari in Mac OS X v10.4 Tiger in 2005. At the time there was
a growing realization among computer users that web browsers
retained many kinds of information about websites that were
visited, and that this information could be damaging to a person's
privacy in the case of a shared computer. Although web browsers had
the ability to ``clear history,'' ``clear cookies'' and ``empty the cache,'' these are
typically drastic actions that remove useful information or important
as well. What's more, they leave evidence that something has been
deleted. Apple's ``private browsing'' mode instead let users simply
tell Safari to stop recording what the user was doing.

The idea of private gained popularity and was implemented in all of
the major browsers within a few
years. \citeN{Aggarwal:2010:APB:1929820.1929828} analyzed the private
browsing models of several popular browsers and found that they all
implemented the feature differently and retained different kinds of
information during private browsing. This isn't entirely surprising,
of course, since there are many ways that a website can leave
identifying information on a user's computer, there was no generally
accepted specification as to what private browsing should do, and
there was no way to consistently test a browser to see if it was in
compliance with the user's expectations.

As first blush, testing the private browsing mode of a browser seems
pretty straightforward:

\begin{enumerate}
\item Turn on private browsing.
\item Browse some websites.
\item Analyze the browser and see if any evidence of the websites was
  left behind. 
\end{enumerate}

This simple methodology has an important problem. Without testing and
experimentation, we have no way of knowing if our tools can find the
kind of information left behind during the course of normal
browsing. As a result, if we don't find information, we don't know if
the emptiness is the result of a properly implemented private browsing
mode or the result of tool failure.

A second problem with this methodology is that it doesn't give us
enough information for explaining the information we do find. That is,
there's no way to know if the information found is the result of our
browsing activities or something else.

\subsection{Experiment \#1: Understanding Normal Behavior}

Before we can evaluate the private browsing mode, we need to be able
to characterize the browser's normal behavior. Specifically we need to
verify that we can reliably detect information that we place on the computer
during normal browsing---we need to test our acquisition and analysis
steps to make sure that they work as expected. We can perform such a test with this sequence of steps:

\newcounter{protocol}\setcounter{protocol}{0}
\newcommand{\protocol}[#1]{\noindent\textit{Protocol \#\arabic{protocol}: #1\refstepcounter{protocol}}}

\protocol{Normal Behavior, first try}
\begin{enumerate}
\item Be sure that private browsing is turned off.
\item Browse some websites.
\item Analyze the browser with the DF tool to see if any evidence that the websites
  were visited was left behind. 
\end{enumerate}

This is better, but there is still an important problem: we don't know
if the data was placed there as a result of our browsing.

\subsubsection{Documenting the Initial State}
In order to know that our browsing was responsible for the evidence
that was recovered, we need to verify that the evidence was not
present \emph{before} we performed our browsing. To do that we need to
document the system's state before our experiment
begins, the \emph{initial state}. We will perform this analysis with
the same DF tool. Instead of analyzing the system once, we'll analyze it
twice---before and after the intervention---and compare the
results. Anything that's changed should be the result of our activity.

This expands our sequence of steps, which is now complex enough that
we need to introduce some notation:


\protocol{Documenting initial state}
\begin{enumerate}
\item Be sure that private browsing is turned off.
\item Using the DF tool, capture the system's state and store it in dataset $D_0$.
\item Browse some websites.
\item Capture the system's state a second time and store the results in dataset $D_1$.
\item Compare the datasets $D_0$ and $D_1$ to determine what's
  changed. Analyze this changed data to see if they contain any
  evidence of the browsed websites. 
\end{enumerate}

\subsubsection{Clearing the system}

It is easier to analyze systems that are simple. A problem with the
approach above is that the system is likely to have extraneous data and
programs that is not relevant to our inquiry. We will respond by
\emph{clearing}\footnote{``Clearing information is a level of media sanitization that would protect the confidentiality of information against a robust keyboard attack. Simple deletion of items would not suffice for 
clearing. Clearing must not allow information to be retrieved by data, disk, or file recovery 
utilities. It must be resistant to keystroke recovery attempts
executed from standard input devices and from data scavenging tools. For example, overwriting is an acceptable method 
for clearing media. 

There are overwriting software or hardware products to overwrite storage space on the 
media with non-sensitive data. This process may include overwriting not only the logical 
storage location of a file(s) (e.g., file allocation table) but also may include all addressable 
locations. The security goal of the overwriting process is to replace written data with 
random data. Overwriting cannot be used for media that are damaged or not writeable. 
The media type and size may also influence whether overwriting is a suitable sanitization 
method. [SP 800-36]. 

Studies have shown that most of today’s media can be effectively cleared by one overwrite.''\cite{nist-800-88}.
the system before we begin the
experiment. This adds two more steps to our procedure:

\protocol{Clearing the system}
\begin{enumerate}
\item Clear the computer's hard drive.
\item Install and operating system and a web browser.
\item Be sure that private browsing is turned off.
\item Using the DF tool, capture the system's state and store it in dataset $D_0$.
\item Browse some websites.
\item Capture the system's state a second time and store the results in dataset $D_1$.
\item Compare the datasets $D_0$ and $D_1$ to determine what's
  changed. Analyze the changed data to see if they contain any
  evidence of the browsed websites. 
\end{enumerate}


\subsubsection{Creating Self-Identifying Data for Recovery}
At this point we should add some more structure to the task that we
will be performing with the browser. Instead of browsing ``some
websites,'' our protocol should precisely specify which websites will
be browsed and what kinds of data we hope to recover. 

We'll address this issue by visiting specific URLs on specific
websites that are created explicitly for the purpose of this
experiment and do not previously exist on the planet. 
An easy way to do this is to use a combination of letters and
numbers that do not appear on any computer and cannot be found by
Internet search engines. We will call this string the experimental
token. There are many such examples---here we will
use exp201990176, although you should use your own. Having chosen and
tested the sequence, you can be pretty sure that the sequence does not
appear in a subdomain (e.g. \texttt{page201990176.nps.edu}) or a domain
  within a common top-level domain (e.g. \textt{exp201990176.org}).

In a typical experiment there is frequently the need for multiple
tokens. For example, we may wish to have one for a domain
(e.g. \url{http://exp201990176.org}), one for the base of the web
server's pages
(e.g. \url{http://exp201990176.org}/page201990176.html), and one for
the contents of the pages (e.g. ``This is our token:
  content201990176'').  We call these kinds of tokens
  ``self-identifying'' because they uniquely identify both the
  experiment and their data contents. All of these tokens are based on
  the \emph{key} 201990176. We use the phrase
  \emph{collateral} to describe this site. It should be hosted on a
 computer other than the system under study, of course.

After we run our experiments and collect our data, we will search for
the specific tokens or for the key in the data that we recover. Not
finding a token is not a guarantee that it is not present---the token
may have been encoded or encrypted or split into pieces that are
stored disjointedly. But since it is easier to find residual data if
you know what you are looking for, self-identifying data can help a
lot.

\protocol{Background and Self-Identifying Data}\label{protocol:background}
\begin{enumerate}
\item Create three tokens and the corresponding Internet collateral.
\item Clear the test computer's hard drive.
\item Install and operating system and a web browser.
\item Be sure that private browsing is turned off.
\item Using the DF tool, capture the system's state and store it in dataset $D_0$.
\item Browse the collateral website
 and the specific pages on it.
\item Capture the system's state a second time and store the results in dataset $D_1$.
\item Compare the datasets $D_0$ and $D_1$ to determine what's
  changed. Analyze the changed data to see if they contain any
  evidence of the browsed websites. 
\end{enumerate}


Although self-identifying data helps, at some point you also need to
think about the specific mechanisms that may result in data being
captured. Browsers retain data differently if the URL is typed
into the search bar or found by clicking on a link. Here you can use
self-identifying data again, by creating different tokens for each
specific kind of transition or activity that you wish to capture. 

Unfortunately, you cannot rely only on self-identifying data: in order
to create a collateral website that's truly realistic, the website
needs to replicate all of the functionality on actual
websites. Replicating this functionality isn't hard for sites like
Wikipedia.org or Wordpress.org, where the underlying software is open
source. It's quite difficult for major news sites or search engines,
many of which rely on JavaScript running inside web browsers to
construct an experience with information served from many different
web servers, typically operated by many different organizations.

For these reasons, privacy auditing experiments need to use combine
self-identifying data with real-world data. However, for our purposes
here, working with self-identifying data is sufficient.

\subsubsection{Being specific}
For our final modification involving turning it from a rough sketch of
the experiment to a step-by-step protocol that we can repeatably
follow. Protocol~\ref{protocol:backgroudn} leaves a lot of details
unspecified. That's okay for a planning document, but those details
must be filled in before the experiment can actually be
conducted. At some point the experimenter needs to decide which
program to use and what to type at the keyboard. Before you start, it's always 
better to make these decisions in advance, to write them down, and to
show them to someone else for review.

With that in mind, here is a more specific experimental protocol:

\protocol{Final Protocol}
\item Create the Collateral website
\begin{enumerate}
\item This protocol uses the tag key 201990176 and the domain
  dom201990176.m57.biz, a domain we host commercially.
\item Create the pages page201990176-1.html, page201990176-2.html and
  page201990176-3.html with the contents as indicated by table~\ref{tab-contents}.
\end{enumerate}
\item Run the experiment
\begin{enumerate}
\item Start with a laptop computer with a 60GB hard drive.
\item Book a copy of Ubuntu Linux with a Live CD.
\item Open a Terminal. 
\item Determine the device associated with the internal hard drive
  (it's probably \verb+/dev/sda+)
\item Clear the internal hard drive with the \texttt{dd} command:\\
      \verb+# dd if=/dev/zero of=/dev/sda bs=65536
\item Close the Terminal
\item Partition and install Ubuntu Linux to the laptop's internal
  drive using the Ubuntu installer.
\item Create the account ``user'' with a password ``password''.
\item Reboot the laptop.
\item Log into the ``user'' account.
\item Open the Terminal
\item Launch the web browser that Ubuntu installed.
\item Go to the page \url{http://www.wikipedia.org}.
\item Close the web browser's window to exit the web browser.
\item Click on the Terminal window's title bar and type the command:\\
     \verb|$ tar cfvz /tmp/data0.tar.gz $HOME|
\item Open the web browser again.
\item Enter the three URLs from Table~\ref{tab-contents} into the
  Browser's address bar, waiting for each page to load.
\item Close the browser.
\item Click on the Terminal window's title bar and type the command:\\
     \verb|$ tar cfvz /tmp/data1.tar.gz $HOME|
\item Copy \verb+/tmp/data0.tar.gz+ and \verb+/tmp/data1.tar.gz+ to
  another computer for analysis.
\end{enumerate}

\begin{table}
\begin{tabular}{lll}
tag & URL  \\
    & contents \\
\hline
contents201990176-1 & \url{http://dom201990176.m57.biz/page201990176-1.html} \\
                    & \verb+<html><body>contents201990176-1</html></body>+\\
\hline
contents201990176-2 & \url{http://dom201990176.m57.biz/page201990176-2.html} \\
                    & \verb+<html><body>contents201990176-2</html></body>+\\
\hline
contents201990176-3 & \url{http://dom201990176.m57.biz/page201990176-3.html} \\
                    & \verb+<html><body>contents201990176-3</html></body>+\\
\hline
\end{tabular}
\caption{Contents of the collateral website for the first experiment}\label{tab-contets}
\end{table}

\subsubsection{Background Behavior}

It may be tempting to simply visit popular news and social media
websites. The problem here is that there may be background processes
or programs on the computer that also visit these sites. We certainly
want to visit these sites, because they may employ specific techniques
that are not elsewhere, but we need to further extend our experiment
so that we can distinguish changes that result from our activity from
those that result from the system's natural activity.




To make our task easier, one kind of data that we'll try to recover
will be ``self-identifying.'' That is, the data will be labeled so
that we know that i

There are many ways that websites can leave evidence of their being
visited in a browser: in the history list, in cookies, the cache,
bookmark list. 
and 

We'll stop here, but it should be clear that we can do far more.


---it requires a series of experiments 



So in practice we might want a more complex procedure:

\begin{enumerate}
\item We will analyze the system with our tools to find out what
  information is present. We will call this dataset $D_1$. 
\item We will use a digital system.
\item We will analyze the system a second time with our tools,
  producing dataset $D_2$. 
\item We will compare $D_2$ with $D_1$ and determine what changed. In
  particular, we will look to see if privacy-sensitive information was
  present in $D_2$ that was not present in $D_1$. 
\end{enumerate}



and
the tools that we use to analyze them are complex systems. In practice
we need to understand both systems are
complex systems. , and digital forensic tools rarely provide the
exact answers for which their users are looking. kind of information that we're looking for. 

in
digital systems. But experimentation plays many other roles as well:



\begin{enumerate}
\item We test digital forensics tools to see what kind of information they can recover.
\item We test the information that is recovered to see if it
  accurately reflects the information that was originally present.
\item We test computer systems to see what kinds of information they have.
\item We test software that runs on computer systems to see what kinds
  of information it may create, destroy, or leave behind.
\end{enumerate}

As with cooking, digital forensics experiments can result in epistemological
confusion---we might think that the experiment means one thing when in
fact it means something else. If a tool finds a piece of data on a piece of media, was
the data present or did the tool manufacturer the data? If a tool
fails to find data on media, is the data really not present, or does
the tool have a bug?

There are many ways to reduce the confusion. For example, we can use
tools that have been
tested and validated by a third party.\footnote{The \emph{first party}
  is the tool provider and the \emph{second party} is the tool user,
  so another organization that is testing the tool is said to be a
  \emph{third party}.} The Computer Forensics Tool
Testing Program at the National Institute of
Standards and Technology tests disk imaging tools to make sure that
the tools accurately copy data from a hard drive to a disk image
file. To test the tools, a technician will copy known data onto a hard
drive and then use the tool under test to copy the data off. The copied data
should match the original.

Here's where things get complicated. If the data match, the tool is
not necessarily flawless. And if the data doesn't match, the tool
is not necessarily flawed? The two data sets may match by chance even
if the tool is flawed, and if the data sets may be different for
reasons that have nothing to do with the tool---for example, the drive
may have a bad sector. 

One of the difficult things about digital forensics is that we
can never really trust our tools and what we think they are telling
us. We resolve this issue by designing our experiments with internal
checks and controls. Such measures provide assurance that is mutually reinforcing.

\section{Designing An Experiment}


\subsection{Purpose of your experiment}

Many people start designing an experiment by thinking about what they
are going to do---what materials they are going to work with, what they are going to
change, and what data they are going to collect. This is almost always
the wrong way to design an experiment.

\subsection{Experimental Complexity}

\subsection{Controls}

\subsection{Variables}

\section{Why experiment?}

why should you experiment?

\section{What is the purpose of the experiment? - what you can provide and what you can't}
\section{Start with a wipe}
\section{Use Self-Identifying Data}

What self-identifing data is.


% M-sequences in radar
% KW37 - keying is for a day; you can join anytime.
% Watermarking - covert and robust go together
% hackmem search algorithm

\section{Sampling vs. Complete Analysis}

sometimes you can analyze all the data, but frequently you can't

\section{Working with large amounts of data}
 - What's large?
 - 4GiB limitations (FAT32, ZIP vs. ZIP64)
\section{Error Rates}
\subsection{What is an error rate}
\subsection{Error rates from hardware}
\subsection{Error rates from sampling}

