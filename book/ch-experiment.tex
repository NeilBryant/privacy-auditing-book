\chapter{Designing a digital forensics experiment}

We use experiments to learn something new or to confirm something that
we've learned before. Experiments typically combine some kind of \emph{test}
or \emph{intervention} with one or more \emph{measurements} or
\emph{observations}. 

Experiments are common in digital forensics:

\begin{itemize}
\item We test digital forensics tools to see what kind of information they can recover.
\item We test the information that is recovered to see if it
  accurately reflects the information that was originally present.
\item We test computer systems to see what kinds of information they have.
\item We test software that runs on computer systems to see what kinds
  of information it may create, destroy, or leave behind.
\end{itemize}

These many uses of experiments can result in epistemological
confusion. If a tool finds a piece of data on a piece of media, was
the data present or did the tool manufacturer the data? If a tool
fails to find data on media, is the data really not present, or does
the tool have a bug?

One way to resolving the confusion is to use tools that have been
tested and validated by a third party.\footnote{The \emph{first party}
  is the tool provider and the \emph{second party} is the tool user,
  so another organization that is testing the tool is said to be a
  \emph{third party}.} For example, the Computer Forensics Tool
Testing Program at the National Institute of
Standards and Technology tests disk imaging tools to make sure that
the tools accurately copy data from a hard drive to a disk image
file. To test the tools, a technician will copy known data onto a hard
drive and then use the tool under test to copy the data off. The copied data
should match the original.

Here's where things get complicated. If the data matches, the tool is
not necessarily flawless. And if the data doesn't match, the tool
is not necessarily flawed? The two data sets may match by chance even
if the tool is flawed, and if the data sets may be different for
reasons that have nothing to do with the tool---for example, the drive
may have a bad sector. 

This chapter explores what is involved in creating a digital forensics experiment.

\section{Why experiment?}

why should you experiment?

\section{What is the purpose of the experiment? - what you can provide and what you can't}
\section{Start with a wipe}
\section{Use Self-Identifying Data}

What self-identifing data is.


% M-sequences in radar
% KW37 - keying is for a day; you can join anytime.
% Watermarking - covert and robust go together
% hackmem search algorithm

\section{Sampling vs. Complete Analysis}

sometimes you can analyze all the data, but frequently you can't

\section{Working with large amounts of data}
 - What's large?
 - 4GiB limitations (FAT32, ZIP vs. ZIP64)
\section{Error Rates}
\subsection{What is an error rate}
\subsection{Error rates from hardware}
\subsection{Error rates from sampling}

