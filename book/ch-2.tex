\chapter{Investigation Models}
\setlength{\epigraphwidth}{3in}
\epigraph{By failing to prepare, you are preparing to fail}{Benjamin Franklin}

This chapter continues the introduction of technical privacy
auditing by introducing the Technical Privacy Auditing Model and
previewing three examples of how the model will be applied to
different kinds of data later in this book.

\section{The Technical Privacy Auditing Model}

\begin{enumerate}
\item \textbf{TARGET} Determine the systems or data objects to examine.
\item \textbf{RESEARCH} Find information about the data formats in
  questions and the tools that are available to analyze them. 
\item \textbf{COLLECT} Obtain exemplars, ideally from multiple instances.
\item \textbf{ANALYZE} Extract Features; Look for oddities and outliers
\item \textbf{EXPERIMENT} Show that you understand what is happening
\item \textbf{REPLICATE} Ideally on another system
\item \textbf{REPORT} Share results in a concise \& understandable form
\end{enumerate}


\subsection{Applying the model redacted PDFs}

\subsection{Applying the model to JPEGs}

\subsection{Applying the model to network traffic}

\section{The Digital Forensics Model}

Even though every case is different, DF practitioners have developed an
approach to conducting investigations called the 
\emph{digital forensics model}\cite{pollitt:models}. The 
elements of the model include:

\begin{description}
\item[Preparation] Before performing an investigation, those involved
  must prepare themselves by deciding upon the standards and
  procedures that will be followed; receiving the necessary training;
  and obtaining the appropriate equipment and software required for
  investigations. 

  To use the example of the cell phone from above,
  preparation may involve obtaining specialized hardware that can extract the
  memory from a cell phone and practicing one's technique on phones that are not part
  of the investigation.

\item[Collection and Preservation]
  Before data can be analyzed, they   must be removed from the
  device being analyzed and preserved to create a lasting
  record. Without this step, it may not be possible to repeat an
  analysis at a later point in time. 

  In our example, this step might involve the actual extraction of data
  from the cell phone into a single file called a \emph{physical
    image} that records all of the phone's applications, phone data, user data,
  and deleted files. Such a physical image might be 64GB in size or
  even larger.

\item[Examination and Extraction] Working with the preserved data, an
  examiner will explore for any information that might be
  relevant to the investigation at hand. Once that information is
  found, it is extracted and isolated.

  In our example, this step might involve the extraction of each
  digital photograph from the \emph{physical image} and the storage of
  each photograph in its own file. The files might be named according
  to where they were found in the physical image (e.g. 62071808.jpg)
  rather than with the name that they were given by the phone
  (e.g. IMG001.JPG). The examiner might also extract \emph{metadata}
  from each photograph such as the time and GPS coordinates associated
  with each exposure and the serial number of the camera that took the
  photograph. This information might be stored in a separate file for
  each photograph (e.g. 62071808.txt) or might be stored in a single
  spreadsheet for all of the photographs (e.g. jpeg-metadata.xls).

\item[Analysis] Once the specific data being analyzed has been
  extracted, the analyst will construct one or more hypotheses that
  uses the digital evidence to explain possible past activities. A
  hypothesis may draw from multiple digital devices---an email message
  sent from a desktop computer to a cell phone, for example---or the
  hypothesis may incorporate events in the physical world, like a
  power failure or theft. During this phase a good analyst will also
  try to construct alternative hypotheses that are consistent with the
  evidence but which point to different conclusions. 

  In our example, the hypothesis may be that a specific photograph was
  taken with the phone in question. This may be supported by the
  photograph having similar metadata to a photograph taken a few
  minutes earlier or later. An alternative hypothesis may be that the
  photograph was downloaded to the phone over a network.

\item[Reporting and Testimony] Finally, the analyst will
  produce a written report or give testimony in a courtroom.
  Judges and juries can't examine digital evidence for
  themselves---even if they had the training and the technical skills
  to do so, performing their own analysis would be inappropriate:
  their role in the legal process is to evaluate the law, the evidence
  and make a legal determination, not to perform technical
  analysis. Reports and testimony must therefore be 
  \emph{complete}---they must describe the tools and procedures
  that were followed, clearly document what was found, and then
  separately provide the technical interpretation of the
  evidence. 

  In our example, the analyst might prepare a report documenting how
  the phone's contents were copied, the tools that were used to
  extract the photographs and their corresponding metadata, and how
  the evidence is consistent with the analyst's hypothesis. 

\end{description}

The model brings reliability and repeatability to the process, helping
to assure that different examiners working with the same data will
arrive at the same conclusion. 

Because they can look into the past and uncover hidden data, DF tools
are increasingly used beyond the courtroom. Security professionals use
DF tools to analyze network intrusions---not to convict the attacker,
but to understand how the attacker gained access to plug the
hole. Data recovery firms use DF tools to resurrect files from drives
that have been inadvertently formatted or damaged. 

