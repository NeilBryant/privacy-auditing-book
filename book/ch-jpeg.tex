\chapter{Privacy Leaks with Digital Photographs}

It is hard to understand the impact that low cost digital photography has had on our
planet. Just two decades ago, the act of making a picture was a
relatively rare and special event. Most people did not carry
cameras---those people who did frequently stood out. Cameras were
frequently viewed with suspicion. Most events were not visually
recorded. 

Today matters are reversed in much of the world. Much of the
industrialized world now has more activated smartphones and feature
phones than human
beings\footnote{\url{http://www.ctia.org/advocacy/research/index.cfm/aid/10323}}\footnote{\url{http://en.wikipedia.org/wiki/Mobile_phone_penetration_rate}},
and even low-cost modern feature phones have digital cameras. Many
events are recorded---sometimes covertly---and those recordings can be
redistributed to millions of people with ease. Photographs and videos
made by bystanders have had profound impact on many world events.

This chapter is not about the ability of digital photographs and
videos to violate privacy based on their overt content. Instead, this chapter
is about the ability of digital media to inadvertently reveal
private information without the realization of the photographer or
publisher.

\sgraphic{ch-jpeg}{Reprinted with permission from \url{http://1235}}

\section{The JPEG file format}

\section{Thumbnails}

\section{Exif metadata}
\subsection{GPS Coordinates}
\subseciton{Date and Time}
\subsection{Other Exif Information}

\section{Non-obvious image content information}
\subsection{Non-Subject Information Leakage}
\subsection{Reflections}
\subsection{Details in Shadows}
\subsection{High-Resolution Information}




Today cameras are widespread.
