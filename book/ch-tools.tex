\chapter{Open Source Forensic Tools}

\section{Sleuth Kit}

Sleuth Kit is an open source digital forensics toolkit for extracting
files from disk images. Sleuth Kit understands a variety of disk image
formats, partitioning schemes, and file systems. With Sleuth Kit, you can can recover
both allocated files and files that have been deleted directly from a
disk image, without having to ``mount'' the disk image by the host
operating system. An added advantage of Sleuth Kit is that it runs on
Windows, Linux, and Macintosh systems, allowing you to access data
from disk images even if your underlying operating system does not
understand the format.

The Sleuth Kit website\footnote{http://sleuthkit.org} has information
on Sleuth Kit, Autopsy (a graphical interface for SleuthKit), the
Sleuth Kit Hadoop Framework (for processing large numbers of hard
drives in a cloud computing environment), and other tools. 
Precompiled binaries for

\subsection{Sleuth Kit under Windows}
Windows users should download pre-compiled Sleuth Kit binaries from Source
Forge.\footnote{\url{http://sourceforge.net/projects/sleuthkit/files/sleuthkit/4.0.2/}}
and install them in into |c:\sleuthkit|.

\subsection{Sleuth Kit under Linux}

\subsection{Sleuth Kit under Macintosh}

\section{Network Monitoring with WireShark and tcpflow}

\subsection{WireShark and tcpflow on Windows}
\subsection{WireShark and tcpflow on Linux}
\subsection{WireShark and tcpflow on Macintosh}

\section{Bulk Data Analysis with Bulk Extractor}

\subsubsection{Bulk Extractor on Windows}
\subsubsection{Bulk Extractor on Linux}
\subsubsection{Bulk Extractor on Macintosh}

\sgraphic{ch-1/windows-console}{Setting your Windows console for 132
  rows of text and 9999 lines of scrollback will improve the usability
  of many text-based commands.}

\section{Text Editors}\label{sec:text-editors}

\section{PostScript and PDF Viewers}


\section{Making Hex Dumps}

Notepad++ / Install a hex editor plugin

\subsection{Linux}
On Linux systems you may need to explicitly install Python 3. Do so on
Fedora by typing:

\begin{code}
$ (@ \hl{sudo yum install python3} @) 
\end{code} 
% $

\subsubsection{Hex Dumps under Linux (Fedora)}

Popular tools for creating hex dumps on Linux include \texttt{od}
(octal dump) and \texttt{xxd}.  The \texttt{od} is part of the base
release. xxd must be installed from the package
\texttt{vim-common} with the yum command:

\begin{code}
$ (@ \hl{sudo yum install vim-common} @)
\end{code}

Note: We're not sure why xxd is part of
  vim-common. If you didn't know this, you could could the command
  \texttt{yum whatprovides} to learn which package provides it:
\begin{code}
$ (@ \hl{yum whatprovides xxd} @)
Loaded plugins: langpacks, presto, refresh-packagekit
2:vim-common-7.3.712-1.fc18.x86_64 : The common files needed by any version of the VIM editor
Repo        : fedora
Matched from:
Filename    : /usr/bin/xxd
$
\end{code} 



\subsection{MacOS}

