\section{Terminology}
\begin{description}
\item[Allocated (files and sectors)] Allocated files are files that can be viewed through
  the operating system; allocated blocks are blocks that have been
  allocated to a specific file and will not be re-used by the
  operating system unless the contents are first relocated
  elsewhere. cf with \emph{deleted files}.
\item[Bit] A binary digit (0 or 1). The word ``bit'' was coined by
  John W.\ Turkey (1915-2000) while working at Bell Labs and first appeared in
  print in Claude Shannon's seminal 1948 article on information
  theory, \emph{A Mathematical Theory of Communication}. (Turkey also
  coined the word \emph{software}.) Abbreviated with a lower-case
  letter \emph{b}.
\item[Byte] An ordered set of bits used to represent the minimum
  amount of data that can be read or written to a computer memory at a
  time. Modern computers use 8-bit bytes, allowing for 256
  ($2^8$)possible values (typically 0 through 255 if the byte is
  interpreted as an unsigned number, or -128 through 127 if the byte
  is interpreted as a signed value.) Abbreviated with an upper-case
  letter \emph{B}.
\item[Compression] A mathematical process for reducing the size of a
  file or other digital object. Compression can be \emph{lossless} or
  \emph{lossy}. In Lossless compression the original file can be recovered by
  \emph{decompressing}, while with lossy compression information is
  lost and the exact original cannot be recovered. (However, lossless
  compression may have such high fidelity that a human being is unable
  to distinguish between the original and the decompressed objects.
\item[Deleted files] Files whose contents can be recovered but the
  sectors of which are not allocated.
\item[Digital Forensics] ``A branch of forensic science encompassing the
  recovery of and investigation of material
  found in digital devices, often in relation to computer crime.''\cite{reith:examination}
\item[Digital Trace Evidence] Another name for \emph{residual data}.
\item[Disk Image] A byte-for-byte copy of all the data on hard drive,
  camera card, or other kind of sector-oriented mass storage
  device. Disk images can contain additional metadata such as an
  embedded cryptographic hash of the original media, the date that the
  image was made, the examiner who made the image, as well as notes.
\item[Disk Imager] A program for making disk images.
\item[Media Exploitation] The extraction, translation and
  analysis of digital documents and media to generate
  useful and timely information.
\item[File] An ordered collection of 0 or more bytes. Files have a
  length; they may optionally metadata such one or more file names,
  modification dates, owners, and other information.
\item[File Header] One or more fixed bytes at the beginning of a
  file. For example, ZIP files begin with the file header ``PK'' (hex
  \texttt{50 4B}), while JPEG files begin with the field header
  \texttt{FF D8 FF E0}.
\item[File Footer] One or more fixed bytes at the end of a file. JPEG
  files end with the file footer \texttt{FF D9}.
\item[File System] Part of a computer's operating system which
  controls the storage of files on a mass storage system such as a
  hard drive or camera card.
\item[Forensic Science]
\item[Hash Value] Always say \emph{hash value}, never \emph{hash
  code}, since the word \emph{code} might refer to either the hash
  value or the program that computes the value.
\item[Investigator-Centric] Digital Forensics is said to be
  ``investigator-centric,'' meaning that most of the advances in the
  field have been to serve specific needs of investigators, rather
  than based on what is scientifically or technically possible.
\item[Logical Block Address (LBA)] \emph{Sectors} of a mass storage
  device are arranged in numbered sectors, starting with LBA 0. Early
  PC systems used hard drives that supported a 22-bit LBA and 512-byte sectors, allowing for a
  total of $2^{22}\times 2^{9}=2^{31}=2\textrm{TB}$ of storage. The
  current Advanced Technology Attachment standard, ATA-6, supports
  48-bit LBA addresses and a 4096-bit sector size, for a maximum disk
  with $2^{48}\times 2^{12}=2^{60}\approx 1,152,921\textrm{TB} \approx 1
  \textrm{EB (exabyte)}$ of storage.
\item[Malware] Software that embodies evil intent, such as to damage
  computer systems or steal private information. Computer Viruses,
  worms, and Trojan horses are examples of malware. 
\item[Metadata] Data about other data. The created, modified and
  access timestamps associated with a file on a camera card are
  examples of metadata.
\item[MD5] Message Digest \#5, a cryptographic hash algorithm
  developed by MIT professor Ron Rivest in 1991. Although MD5 is
  widely used in computer forensics, MD5 has known flaws and should
  not be used in applications that depend upon collision resistance.
 \item[Network forensic analysis tool (NFAT)] A system that can record
   packets as they move over a network and perform detailed
  after-the-fact analysis.
 \item[normalization]
 \item[Packet] A set of bytes that are sent over a network. Most
   Internet packets range in size from 40--1500 bytes.
 \item[PCAP File] A packet capture file, which is typically a set of
   packets that were recorded using a packet sniffer.
 \item[Packet sniffer] A program or device that can record packets are
   they move over a network.
 \item[Packet Traces]
 \item[Provenance]
\item[Preview] an application on Apple Macintosh for viewing images
  and PDF files. 
\item[Residual Data] data that is left behind on a computer after an
  operation is completed but is no longer in active use. For example,
  most computer systems erase the file name when a file is deleted,
  but do not overwrite the actual file contents. These contents remain
  on the drive as residual data, recoverable with computer forensic tools.
\item[Sector] The minimum amount of data that can be read or written
  to a mass storage device. Modern hard drives use sectors of 4096
  bytes; drives with 512-byte sectors have been common since the mid
  1970s and are still widely used today.
\item[Subject Computer and Data] The computer system and data that are
  being analyzed as part of a digital investigation. This may be data
  extracted from a computer belonging to a suspect in a crime, but it
  may also be data from the computer of a victim. Subject data may
  even be data generated during the course of a DF tool.
\item[Subject] is a common shorthand for the owner or primary use of a computer
  system from which subject data was obtained. 
\item[Tool-Based] Digital Forensics is said to be ``tool-based,''
  meaning that most investigators in their investigations to the
  capabilities provided by today's tools---investigators generally do
  not devise their own digital experiments or invent new approaches
  for working with data to solve specific cases.
\item[Timestamp]
\item[Unallocated sectors] Sectors that are not allocated to files or
  metadata, but are instead available for use by the file system
\end{description}
