\chapter{How Data are Stored}
As indicated above, while the primary storage system of most computers can be accessed a
byte at a time, secondary storage can only be accessed in
blocks. Secondary storage, in turn, can be divided into two broad
categories:

\begin{description}
\item[Block addressable] storage systems are systems for which any
  block can be individually read or written. Each block in a block
  addressable system is assigned a \emph{logical block address}
  (LBA). 
\item[Streaming] storage systems are those that allow blocks to be
  read or written as part of a sequence. Most streaming systems are
  based on a spool of magnetic or optical tape. New blocks can be
  written to the tape and the tape supports limited operations---for
  example, \emph{rewind}, writing a \emph{mark}, \emph{seeking} to a mark,
   \emph{erasing} and possibly \emph{overwriting}. In general it is
   not possible with a streaming system to reliably overwrite a single
   block---attempts to do so may result in other blocks being damaged
   and rendered unreadable.
\end{description}

Confusingly, although we have used the term \emph{block} exclusively
until now, many vendors use the term \emph{sector} to describe blocks
that are written to spinning discs. (The term is presumably comes from
the fact that data was originally stored in concentric circles on a spinning platter
and in bands on a rotating magnetic drum. Thus, each block occupied a
sector of a circle.) Until the mid 2000s most mass storage systems
used blocks (sectors) that were 512 bytes in size. Since then vendors
have moved to systems that use 4096-byte blocks internally, although
many still give the appearance of using 512-byte sectors to provide
for software compatibility. 

\section{Drives, Hard Drives, Solid State Drives}

\section{Write Blockers}

Removable media such as floppy disks, tapes and storage cartridges
used in the 1990s generally had some kind of \emph{write-protect}
switch or tab that could be used to prevent inadvertant alternation or
overwriting of evidence. Hard drives of the time had no such
facility. To overcome this deficiency the industry invented
\emph{write-blockers}, a device that can be inserted inline between a
hard drive and a computer system. A typical write blocker might have a
male and a female ATA-33 connector; the female connector plugs into
the hard drive's male ATA-33 connector, while the blocker's male
connector plugs into a cable that connects to the host computer.

Write blockers allow examination of subject data without fear of
inadvertently modifying the contents---provided that the write blocker
works properly. A problem with the concept of write blockers is that
the only real specification of what these devices should do was their
name---that is, they should block modification of data on the hard
drive.  But there
are two ways to do this. One is to literally block the commands that
alter data, a task that requires a clear enumeration of all such
commands. An alternative (and somewhat safer) strategy is to only pass
those commands that are known \emph{not} to alter
data\cite{dfrws2006:JamesLyle}. Both of these approaches
implicitly assume that the only way data is altered on the drive is
through the execution of commands, and this is not actually the
case. For example, the S.M.A.R.T. counters inside modern drives that
track the number of seconds the drive is powered up will continue to
advance even when a write-blocker is in place.

\subsection{Disk Imaging and Disk Images}
Two problems that are not solved by write blockers. First, the
digital evidence still resides on the original media, which is a
mechanical device and subject to failure. Second, most legal systems
allow for both parties to have access to evidence. Both of these
problems can be overcome by making a sector-for-sector copy of the
disk, a process called \emph{disk imaging}.

The most basic way to image a disk is to copy every sector onto
another disk of the manufacturer and model number. Such a disk is
called a \emph{mirror copy} or \emph{mirror volume}. Once the copy is made, the subject disk
can be kept sealed in an evidence locker and the mirror volume can be
used for forensic analysis. 

A difficulty in making a mirror copy is that it may not be possible to
obtain a drive of the exact make and model number. Fortunately, it is
only necessary for the copy disk to be larger than the original. The
sectors between the end of the original disk and the copy disk should
be ignored.

[Figure: original disk, copy, and the section to ignore.]

\subsection{Inaccessible Information}

TK - DCO and HPA

TK - SMART information

