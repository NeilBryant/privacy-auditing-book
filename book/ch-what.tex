\chapter{What Data Look Like}

Sometimes you can spot data leakage by simply opening a file with an
ordinary utility, as we did in the last chapter when we used Apple's
Preview application to read geolocation data hidden inside a JPEG. 

Other times the leakage may be less obvious.  The software developers
know that the information is present but don't think that it's
relevant, and people who are in a position to know that the leakage is
relevant don't have the technical ability to determine that it's
present.

This chapter starts with the creation of a Portable Document Format
(PDF) file that contains hidden information
(\secref{sec:make-pdf}). Using this file, we'll explore two popular
ways of looking at data it contains---by looking at \emph{hex dumps}
(\secref{sec:hex-dumps}) and by extracting printable strings
(\secref{sec:strings}). 

In order to understand what the hex dump and extracted strings
actually mean, it's important to first understand how numbers
(\secref{sec:numbers}) and letters (\secref{sec:letters}) are actually
stored inside a computer's memory.  Much of this chapter is similar to
what might be covered at the start of a course on computer
architecture. But whereas an architecture course is primarily
concerned with computation and data in motion, this chapter is
concerned with understanding data at rest.

\section{A Leaky File with ``Hidden'' text}\label{sec:make-pdf}

\figref{ch-what/hidden_1} shows the creation of a simple file that
contains hidden data with Apple's TextEdit application. To make the
file we open TextEdit, type the text, select the word ``Hidden'' and
change the background (highlight) color of the text from ``none'' to
``black.'' We save two copies of the file. The first is a so-called
Rich Text Format (RTF) file called |hidden.rtf|. The contents are then
exported Adobe Portable Document Format (PDF) file and saved as |hidden.pdf|.


\bifigure{ch-what/hidden_1}{ch-what/hidden_2}{Constructing a PDF with
  hidden text by selecting the text in a word processor (left) and
  then setting the highlight color to black (right).}

The file |hidden.rtf| is seven lines long and has just 297 bytes; the
contents appear in \lstref{hidden.rtf}. The file |hidden.pdf|, in
contrast, is 8304 bytes long. If the file's contents were printed here
they would fill more than four pages, but the contents would appear as
a confusing jumble of letters, numbers, and obscure symbols---what's
sometimes referred to as \emph{garbage}, and more properly termed
\emph{binary data} to distinguish it from the printable text that
makes up |hidden.rtf|. Attempt to view the file |hidden.pdf| with the
Unix program \emph{less} and you'll be warned:

\begin{code}
$ (@ more hidden.pdf @)
"hidden.pdf" may be a binary file. See it anyway?
\end{code}
%$

Type ``y'' and you'll see results similar to
\figref{ch-what/hidden-more}. The rest of this chapter explains what
this display means.

\sgraphic{ch-what/hidden-more}{The file {\tt hidden.pdf} viewed with the program \emph{less}.}

\lstinputlisting[caption=hidden.rtf,label=hidden.rtf]{ch-what/hidden.rtf}

\section{Bits, Bytes, and Memory}\label{sec:numbers}
The \emph{bit} is the fundamental unit of storage on digital computers. Bits
have two states. Typically we call these |0| and |1|, but they can be any
two values (e.g. true/false, yes/no or up/down). This book 
uses the values |0| and |1| for bits.\marginpar{The word \emph{bit} is
  a contraction of the words \emph{binary digit}.}

An ordered group of 4 bits is called a \emph{nibble}. Each bit can be
a |0| or a |1|, creating 16 possible different values. Note
that $2^4=16$. It is conventional to use the digits 0 through 9 to
number the first 10 possible nibble values, and use the letters A
through F to number the next six, as shown in \figref{nibble}. By
convention computer users call this notation \emph{hexadecimal},
although it's also called Base 16.

To avoid confusion between decimal and binary notation, this book will
always suffix binary numbers with a ``b''. Hexadecimal values have a
``h'' suffix. Some programming languages prefix hexadecimal strings
with a |\x| or |0x|. 

\marginpar{There are 10 kinds of people in the world. Those who know
  binary, and those who don't.}


\begin{figure}
\begin{tabular}{>{\tt}c>{\tt}cc>{\tt}cc}
\textrm{binary} & \textrm{octal} & decimal & \textrm{hexadecimal} & english \\
\hline
0000b & 000 & 0 & 0h & zero \\
0001b & 001 & 1 & 1h & one \\
0010b & 002 & 2 & 2h & two \\
0011b & 003 & 3 & 3h & three \\
0100b & 004 & 4 & 4h & four \\
0101b & 005 & 5 & 5h & five \\
0110b & 006 & 6 & 6h & six \\
0111b & 007 & 7 & 7h & seven \\
1000b & 010 & 8 & 8h & eight \\
1001b & 011 & 9 & 9h & nine \\
1010b & 012 & 10& Ah & ten \\
1011b & 013 & 11& Bh & eleven \\
1100b & 014 & 12& Ch & twelve \\
1101b & 015 & 13& Dh & thirteen \\
1110b & 016 & 14& Eh & fourteen \\
1111b & 017 & 15& Fh & fifteen \\
\hline
\end{tabular}
\caption{A nibble is a set of 4 bits; it has 16 possible values.}\label{nibble}
\end{figure}

Modern computers operate on 8 bits at a time, a unit called a
\emph{byte}. Because each bit in the byte can be a |0| or a |1|, and
because the bits are ordered, the byte can represent the values
\texttt{00000000b} through \texttt{11111111b} --- creating a
total of $2^8=256$ possible different values. This book will normally treat
bytes as unsigned integers in the range 0 to 255, but that is just a
convention; the Java programming language treats bytes as signed
integers in the range $-128$ to 127. It's also common to use present
the value of a byte in hexadecimal notation, with the range |00|
through |FF|.

\subsection{Registers}

Computers use groups of bytes to represent larger numbers. Two bytes
can represent 65,536 different values. Here are some examples:

\begin{tabular}{c>{\tt}c}
decimal & \textrm{binary} \\
\hline
0      & 00000000b 00000000b \\
255    & 00000000b 11111111b \\
256    & 00000001b 00000000b \\
65,280 & 11111111b 00000000b \\
65,535 & 11111111b 11111111b \\
\hline
\end{tabular}

Modern computers use registers in a variety of sizes. Many low-cost
embedded systems use 8-bit and 16-bit registers, most cell phones use
32-bit registers, and modern laptops, desktops and servers use
64-bit registers. There are also 80-bit registers used for some kinds
of scientific computing where both high resolution and high speed are required.

\subsection{Memory}

In the example above, the right-hand byte is called the \emph{least
  significant byte} and represents the bits $2^0$ through $2^7$; the
left-hand byte  called the \emph{most
significant byte}; it represents the binary values $2^8$ through
$2^{15}$. 


Because most computer memories are arranged as an array of bytes,
there are two ways to store these 16-bit values at a particular memory
location. 


% http://en.wikipedia.org/wiki/Single-precision_floating-point_format

Larger numbers Two bytes can be grouped together, allowing values between 0 and
65,535 to be represented. Four bytes


\section{Text, ASCII and Unicode}\label{sec:letters}

In the previous section we looekd 

\section{Hex Dumps}\label{sec:hex-dumps}
A common way to visualize binary data is to format it as a series of
text lines, with each line containing a series of hexadecimal values (one
for each binary byte). Such displays are called \emph{hex
  dumps}.

There are many different programs that produce hex dumps, and each
program formats its output in a slight different manner. Fortunately
there is enough commonality that you can usually look at a hex dump
and figure it out. Most dumps will have a column on the left that
displays the offset in bytes from the beginning of the data, and a
column on the right that shows the ASCII representation of each
byte. Dumps are typically formatted such that each line represents 32
or 64 bytes of data.

Below is a hex dump for the 28-byte string 
|Hello World!\r\nHello World!\r\n| created with the Unix \emph{xxd}
program. Notice that the hexadecimal numbers \emph{do not} have a
``h'' suffix; users are expected to recognize hexadecimal values from
context:

\begin{Verbatim}
0000000: 4865 6c6c 6f20 576f 726c 6421 0d0a 4865  Hello World!..He
0000010: 6c6c 6f20 576f 726c 6421 0d0a            llo World!..
\end{Verbatim}


\section{Strings}\label{sec:strings}

Computer texts use the word \emph{string} inconsistently. Sometimes
the word refers to any sequence of zero or more bytes; sometimes it
means zero or more characters; and sometimes it means zero or
more printable characters. Usually the specific meaning can be
inferred from context. 

One quick way to analyze unfamiliar data is to look for printable
strings, which is the reason that ASCII is shown in most hex
dumps. With large blocks of data it can be hard to find the printable
strings. Consider the file random48.bin:

\begin{Verbatim}
0000000: fdf8 7c33 a153 0516 cc72 84fc 0050 7269  ..|3.S...r...Pri
0000010: 7661 6379 1437 ed4d e84b 0280 a74f e64b  vacy.7.M.K...O.K
0000020: 4695 6a66 9549 2cc9 4109 83c1 3e96 766b  F.jf.I,.A...>.vk
\end{Verbatim}

The file random48.bin contains the word ``Privacy'', but it can be hard to
find. The \emph{strings} program will extract printable
strings from a file and display them on stdout:

\begin{Verbatim}
$ strings random48.bin 
Privacy
$ 
\end{Verbatim}

Because ASCII is commonly used by computer systems, 
\emph{strings} command is a powerful way to find  identifying
information. 

\section{Magic Numbers}

\section{BASE16, BASE64 and BASE85}

\section{Exercises}
Look for private information with strings in files



