The
examples will be based on the two most popular file systems today:
\begin{description}
\item[
\item[NTFS (New Technology File System)] A complex file system developed by Microsoft in
the 1990s for Windows NT and popularized by Windows XP. Today NTFS is
the primary file system used by desktops and laptops running Windows.
\end{description}
\index{File Allocation Table|see {FAT}}
\index{FAT}
\index{Windows 8}
\index{New Technology File System|see {NTFS}}
\index{NTFS}


% The term \emph{physical volume} is used variously to describe a
% physical disk, a partition on a physical disk, or a Logical Unit
% Number (LUN) of a storage system.  Physical volumes can be used
% directly to hold a file system or can be grouped together with a
% Logical Volume Management (LVM) system. An LVM can group multiple
% physical volumes together in to a single physical volume that is large
% or more reliable than the volumes that it built from. Volume managers
% can also implement \emph{snapshots} through the use of
% \emph{copy-on-write}.


When we were finished imaging the card we took it out of the Linux
computer and put it back in the camera. Next we used the camera's
interface to delete some photographs and then took some more. Then we
imaged the card a second time, creating the ``generation 2'' disk
image. This process was repeated a total of six times. By comparing
the different disk images it's possible to understand how the Camera's
firmware manages the files and directories stored on the camera card.

For the remainder of this section we'll work with just the first
generation image, which will call \emph{canon1}. 


