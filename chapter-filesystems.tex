\chapter{Finding Hidden Data in File Systems}
Most kinds of computer data are eventually turned into \emph{files}
and stored on some kind of
\emph{mass storage device} (\figref{art/TK}). At the most basic level a file is
nothing more than a sequence of zero or more bytes. In practice files
have additional properties such as a \emph{file name}, 
\emph{creation date}, \emph{modification date}, \emph{last access
  data}, and so on. These other properties are called \emph{metadata},
a word that loosely means ``data about other data.''

Mass storage systems don't store files. Instead, they store blocks of
bytes (sometimes called \emph{sectors}). These blocks are invariably
sized as an even power of 2. In the 1990s most hard drives had a block
size of 512 bytes and Compact Discs (CDs) had a block size of 2048
bytes; modern hard drives have a block size of 4096 bytes. A 1~TB
drive in a computer laptop would therefore have approximately 250
million 4096-byte blocks. Each block on the media is referenced
by a distinct number called a \emph{logical block address} (LBA). Most
mass storage systems are \emph{random access}, meaning that the
computer can read or write any block in any order, although it is
invariably faster to read or write blocks multiple blocks at a time in numeric order.

Users don't see blocks of bytes when they take an SD card out of a
camera and try to read it on their laptop: they see folders and
files. The translation between the mass storage system's blocks of
bytes and the files that are shown on the computer's screen is
performed by a piece of code inside the operating system called the
\emph{file system}.  The file system takes the concepts of
\emph{files} and \emph{folders} used by people and application
programs, and maps them to specific operations on blocks of data
stored on the mass storage device. When a program opens a named file
and reads the first few bytes, the file system locates the file's
metadata on the drive, determines the specific block associated with
the start of the file, and reads that block into memory. Mass storage
systems can only read or write complete blocks, so the file system may
also have to buffer data between the application program and the
drive.

\section{Exploring the GEN1 SD Camera Card}
In this section we will explore the contents of an SD camera card
created in 2009 with a Canon PowerShot SD800IS camera. The card is called NPS-2009-CANON2.

\begin{figure}
\caption{A conceptual view of NPS-2009-CANON2\label{nps-2009-canon2-conceptual}}
\end{figure}

To create the data for this section, we started with a 32MB camera
card and cleared every addressable sector by overwriting that sector's
content with NULL bytes. You can do this
easily with the program in \figref{overwrite}.


\lstset{basicstyle=\footnotesize,numbers=left,numberstyle=\tiny}
\lstinputlisting[caption=A simple program to clear a disk.]{fs/clear.py}
\begin{figure}
\begin{lstlisting}
\end{lstlisting}
\caption{}\label{overwrite}
\end{figure}

After the card was cleared, we put it into the PowerShot, initialized
the card, and took 36 photos. We then removed the card from the
camera and placed it in a Linux-based computer, where the SD card
appeared as a virtual disk drive. In order to make a copy of the drive
data it is necessary to know the drive name that the Linux system
assigns. For example, if the drive appears as |/dev/sde|, then the
data can be copied off with this Linux  command:

\begin{Verbatim}
$ sudo dd if=/dev/sde of=nps-2009-canon2-gen1.raw conv=noerror,sync
\end{Verbatim} 
%$ 

The dd command copies blocks of data from the input file (\emph{if=})
to the output file (\emph{of=}). Two conversion options are
specified: \emph{noerror} tells the \emph{dd} command to continue even
if it encounters an error, and \emph{sync} tells the program to keep
the output stream synchronized with the input stream in the event of
an error by writing blocks filled with NULLs.  

When we were finished imaging the card we took it out of the Linux
computer and put it back in the camera. Next we used the camera's
interface to delete some photographs and then took some more. Then we
imaged the card a second time, creating the ``generation 2'' disk
image. This process was repeated a total of six times. By comparing
the different disk images it's possible to understand how the Camera's
firmware manages the files and directories stored on the camera card.

For the remainder of this section we'll work with just the first
generation image, which will call GEN1. You can download that file from: \url{http://digitalcorpora.org/nps-2009-canon2-gen1.raw}.


\figref{nps-2009-canon2-conceptual} shows a conceptual view of
GEN1's storage. The card has a Master Boot Record in its
first sector (block 0) which has two slots: slot 1 is an unallocated
region between blocks 0 and 50 (including, somewhat confusingly, the
master boot record itself). Slot 2 is a DOS FAT16 file system that
occupies blocks 51 through 60799. The FAT file system contains its own
structures, including a \emph{Boot Sector Structure} that describes the
file system, a \emph{root directory}, \emph{subdirectories},
\emph{files}, and finally a \emph{file allocation table} that both
describes which sectors belong to which file and indicates which
sectors are free and available for new files. We'll discuss each of
those in the following section.

\subsection{The Master Boot Record (Partition Table)}

The master Boot Record, more commonly known as the MBR, occupies the first
sector of a hard drive used on most Windows-based computers. When the
Windows computer is turned on the computer's Basic Input/Output
System (BIOS) loads these 512 bytes into the start of the computer's
memory and jumps to location 0. Typically the MBR contains a short
program that loads more sectors from the hard drive into memory and
runs them; this second boot loader loads the operating system and runs
it.

The basic structure of the MBR is described by this table:

\begin{minipage}{\textwidth}
\begin{tabular}{|c|c|c|c|c|}
\multicolumn{5}{c}{\bf Structure of the MBR\footnote{From Wikipedia, \url{http://en.wikipedia.org/wiki/Master_boot_record}}}\\
\hline
\multicolumn{2}{|c|}{\bf Address} & \multicolumn{2}{c|}{}                              & \\
\cline{1-2} \bf Hex & \bf Dec         & \multicolumn{2}{c|}{\multirow{-2}{*}{\bf Description}} & \multirow{-2}{*}{\bf Size in bytes}\\
\hline
+000h & +0   & \multicolumn{2}{c|}{Bootstrap code; extended information} & 446 \\
\cline{3-4}
+1BEh & +446 & Partition entry \#1 && 16 \\
+1CEh & +462 & Partition entry \#2 && 16 \\
+1DEh & +478 & Partition entry \#3 && 16 \\
+1EEh & +494 & Partition entry \#4 && 16 \\
\hline
+1FEh & +510 & 55h & & \\
\cline{1-3}
+1FFh & +511 & AAh & \multirow{-2}{*}{Boot Signature} & \multirow{-2}{*}{2} \\
\hline
\hline
\multicolumn{4}{|r|}{\textbf{Total size: $\bf 446 + (4\times16) + 2 $}} & \textbf{512}\\
\hline
\end{tabular}
\end{minipage}

The basic MBR thus provides for a maximum of four partitions, each
described by a partition entry. The entry record includes a byte
indicating if it is active or not, the start of the partition and the
partition's length.  Originally the start and length were encoded as a 
3-byte CHS (Cylinder, Head, Sector) triplet; later the encoding was
expanded to allow a 4-byte LBA (Logical Block
Address). Because blocks
are assumed to be 512 bytes, traditional MBR partitioning allows for a
maximum media size of $512 \times 2^{32}=2\textrm{TiB}$ in storage.

If you review the Wikipedia page you will see that no less than six
different MBRs have been developed and deployed by manufacturers since
MBR records were introduced by IBM PC DOS 2.0 in 1983. All of these
systems store the bootstrap code at location 0, the first four
partition entries starting at location |1BEh|, and have a boot
signature in the last two blocks of the sector. This consistency is
provided for backwards compatibility, allowing each 
generation of disk utilities and BIOS drivers to make some kind of sense of
whatever data they might find on a mass storage device.  Modern
computers use a new partitioning scheme called the GUID Partition
Table (GPT), but even these systems still include a so-called
``Protective MBR'' that allocates the entire drive to a partition of
type |EEh|. This partition literally protects the GPT disk in the
event that the user attempts to mount it on a system that does not
understand GPT partitioning: instead of inadvertently overwriting the
media, the system will believe that the entire disk is taken up by a
partition that the system does not understand.

The SleuthKit |mmls| command will display the contents of the
partition table of any physical device or disk image:

Exercise 1: Use the SleuthKit to View the contents of the master boot
record:

\begin{Verbatim}
C:\ mmls nps-2009-canon2-gen1.raw
\end{Verbatim}

Exercise 2: Alternatively, you can view the contents of the MBR with a
hex dump:

\begin{Verbatim}
$ xxd -len 512 nps-2009-canon2-gen1.raw
0000000: fa33 c08e d0bc 007c 8bf4 5007 501f fbfc  .3.....|..P.P...
0000010: bf00 06b9 0001 f2a5 ea1d 0600 00be be07  ................
0000020: b304 803c 8074 0e80 3c00 751c 83c6 10fe  ...<.t..<.u.....
0000030: cb75 efcd 188b 148b 4c02 8bee 83c6 10fe  .u......L.......
0000040: cb74 1a80 3c00 74f4 be8b 06ac 3c00 740b  .t..<.t.....<.t.
0000050: 56bb 0700 b40e cd10 5eeb f0eb febf 0500  V.......^.......
0000060: bb00 7cb8 0102 57cd 135f 730c 33c0 cd13  ..|...W.._s.3...
0000070: 4f75 edbe a306 ebd3 bec2 06bf fe7d 813d  Ou...........}.=
0000080: 55aa 75c7 8bf5 ea00 7c00 0049 6e76 616c  U.u.....|..Inval
0000090: 6964 2070 6172 7469 7469 6f6e 2074 6162  id partition tab
00000a0: 6c65 0045 7272 6f72 206c 6f61 6469 6e67  le.Error loading
00000b0: 206f 7065 7261 7469 6e67 2073 7973 7465   operating syste
00000c0: 6d00 4d69 7373 696e 6720 6f70 6572 6174  m.Missing operat
00000d0: 696e 6720 7379 7374 656d 0000 0000 0000  ing system......
00000e0: 0000 0000 0000 0000 0000 0000 0000 0000  ................
00000f0: 0000 0000 0000 0000 0000 0000 0000 0000  ................
0000100: 0000 0000 0000 0000 0000 0000 0000 0000  ................
0000110: 0000 0000 0000 0000 0000 0000 0000 0000  ................
0000120: 0000 0000 0000 0000 0000 0000 0000 0000  ................
0000130: 0000 0000 0000 0000 0000 0000 0000 0000  ................
0000140: 0000 0000 0000 0000 0000 0000 0000 0000  ................
0000150: 0000 0000 0000 0000 0000 0000 0000 0000  ................
0000160: 0000 0000 0000 0000 0000 0000 0000 0000  ................
0000170: 0000 0000 0000 0000 0000 0000 0000 0000  ................
0000180: 0000 0000 0000 0000 0000 0000 0000 0000  ................
0000190: 0000 0000 0000 0000 0000 0000 0000 0000  ................
00001a0: 0000 0000 0000 0000 0000 0000 0000 0000  ................
00001b0: 0000 0000 0000 0000 0000 0000 0000 0001  ................
00001c0: 1400 0403 60da 3300 0000 4ded 0000 0000  ....`.3...M.....
00001d0: 0000 0000 0000 0000 0000 0000 0000 0000  ................
00001e0: 0000 0000 0000 0000 0000 0000 0000 0000  ................
00001f0: 0000 0000 0000 0000 0000 0000 0000 55aa  ..............U.
$ 
\end{Verbatim}

Exercise 3: Finally, you can write a program in Python that will open
up the disk image, read the sector into memory, and then print the
fields:

\lstinputlisting[caption=The start of a python program to display the
  Master Boot Record.]{fs/mbrdecode.py}





The first block of a FAT file system contains a \emph{Boot Sector
  Structure} that describes the file system's parameters, including
the sector size, the cluster size, the number of reserved sectors, the
number of File Allocation Tables, and so on. The layout of this
structure is described in Microsoft's FAT specification; you can also
find it in the SleuthKit file |sleuthkit/tsk3/fs/tsk_fs.h|. The first
few lines of the SleuthKit structure are show in \figref{BSS}.

\begin{figure}
\begin{lstlisting}
/*
 * Boot Sector Structure for TSK_FS_INFO_TYPE_FAT_12,
 * TSK_FS_INFO_TYPE_FAT_16, and TSK_FS_INFO_TYPE_FAT_32
 */
    typedef struct {
        uint8_t f1[3];
        char oemname[8];
        uint8_t ssize[2];       /* sector size in bytes */
        uint8_t csize;          /* cluster size in sectors */
        uint8_t reserved[2];    /* number of reserved sectors for boot sectors */
        uint8_t numfat;         /* Number of FATs */
        uint8_t numroot[2];     /* Number of Root dentries */
        uint8_t sectors16[2];   /* number of sectors in FS */
        uint8_t f2[1];
        uint8_t sectperfat16[2];        /* size of FAT */
        uint8_t f3[4];
        uint8_t prevsect[4];    /* number of sectors before FS partition */
        uint8_t sectors32[4];   /* 32-bit value of number of FS sectors */

        /* The following are different for fat12/fat16 and fat32 */
        ...
\end{lstlisting}
\caption{The first few bytes of the Boot Sector Structure, the first
  sector of a FAT file system. From Sleuthkit's \texttt{tsk\_fs.h}.\label{BSS}}
\end{figure}

The FAT file system's 

Because the FAT file system begins in sector 51, the number ``51''
must be added to all internal references.

occu; A listing of
that camera card appears in Figure Tk. Various sectors on the card are
used to store the root directory, a subdirectory ``DCIM'', and
directory entries for the 35 JPEG
images named XXXX through XXXX. Most of the sectors on the card are
used to store the JPEG images themselves. But the card also contains a
significant amount of data that not visible when the card is
inserted into a laptop: there are deleted JPEGs and deleted directory
entries. Some of the deleted JPEGs exist in their entirity and can be
easily recovered using the proper tools. Others exist as fragments and
can only be recovered with difficulty. 

In this chapter you'll learn why file systems retain information after the
user deletes it, how to find and exploit this kind of residual
information, and how to design computer systems so that such
information is not retained.

Tools used in this chapter: hex viewer;

hex editor

SleuthKit precompiled binaries for
Windows. http://sourceforge.net/projects/sleuthkit/files/sleuthkit/4.0.2/
  Download sleuthkit-win32-4.0.2.zip and uncompress the files into
  |c:\sleuthkit-win32-4.0.2|



\section{Background: File System Structures}
This section describes how file systems are structured on a drive. The
examples will be based on the two most popular file systems today, the
FAT file system developed by Microsoft in the 1980s and widely used in
digital cameras, and the NTFS file system developed by Microsoft in
the 1990s and used for most versions of Windows from Windows NT
through Windows 8.


\subsection{Partitioning and Volume Management}

Although a file system can be stored directly on the drive, with the
file sector of the drive being used to store the first sector of the
file system, this is not commonly done. Instead, the first 1 or 2
sectors are used for a \emph{partition table} that describes the
locations of the file systems. In this way a single device can be used
to simulate multiple logical devices. Two commonly used partitioning
schemes are the \emph{Master Boot Record} and the \emph{GUID Partition
  Table} scheme.\wikipedia{http://en.wikipedia.org/wiki/GUID_Partition_Table}
       \wikipediab{http://en.wikipedia.org/wiki/Master_boot_record}

The term \emph{physical volume} is used variously to describe a
physical disk, a partition on a physical disk, or a Logical Unit
Number (LUN) of a storage
system.\wikipedia{http://en.wikipedia.org/wiki/Logical_Unit_Number}
Physical volumes can be used directly to hold a file system or can be
grouped together with a Logical Volume Management (LVM)
system\wikipedia{http://en.wikipedia.org/wiki/Logical_volume_management}. An
LVM can group multiple physical volumes together in to a single
physical volume that is large or more reliable than the volumes that
it built from. Volume managers can also implement
\emph{snapshots}\wikipedia{http://en.wikipedia.org/wiki/Snapshot_(computer_storage)}
through the use of
\emph{copy-on-write}\wikipedia{http://en.wikipedia.org/wiki/Copy-on-write}.

\subsection{Files and File Systems}
The word \emph{file} is commonly used to describe a sequence of data
that is stored on a computer mass storage system. On modern computers
files are sequences of zero or more \emph{bytes}. Files have a length
that is set at any given point in time, but can be readily
changed. Files can have names and other associated \emph{metadata}
such as timestamps, an owner, and a group. Files have traditionally
been grouped together in
\emph{directories}\wikipedia{http://en.wikipedia.org/wiki/File_directory}. also
called \emph{folders}. 

The phrase \emph{File
  System}\wikipedia{http://en.wikipedia.org/wiki/Computer_file} is
used to describe the part of the computer's operating
system that manages a storage system. However, the term is also used
to describe the very sectors on a mass storage device that contain the
files.

It is important to realize that files are an \emph{abstraction} that
was created for the purpose of managing data. There is nothing
inherent in the design of computers, operating systems or mass storage
devices that requires the use of files. A raw disk can be used as
virtual memory \emph{backing store} or as a storage system for a
database. Instead of stored as a sequence of bytes in a file, data can be stored as
objects that are persisted in memory, or as BLOBs in a database.

It is also important to realize that any computational storage device
or abstraction can be used to virtualize and contain any other storage
device. That is, a file can be stored in a file system, but a file
system can also be created and stored inside a file (this is
essentially what a disk image is). Most operating systems can swap to raw
partitions or to files, but file systems can also be created inside
RAM or virtual memory.

Adding to the complexity of the forensic examiner is
\emph{encryption}, which can be applied at the interface between any
of these abstractions. Encryption can be performed in the storage
device itself (as in the case of an encrypting hard drive), in the
disk driver, in the logical volume management system, in the file
system, or at the application level.

\subsection{File Systems of Forensic Interest}
There are a variety of different kinds of file systems in use on
modern computer systems:
\begin{description}
\item[Disk file systems] organize files and directories on
  block-oriented storage systems. These are of interest to those
  engaged in MEDEX operations. Popular file systems include FAT32
  (used primary on removable storage devices and camera cards), NTFS
  (Microsoft's New Technology File System), and HFS+ (Apple's
  Hierarchical File System used on Macs and iPhones).
\item[Distributed file systems] allow a computer to access
  information on remote servers as if it is stored
  locally. Distributed file systems are forensically interesting
  because many use local storage to cache information from the remote
  servers. Analyzing local storage can therefore give clues as to what
  was accessed remotely, and when it was accessed.
\item[Virtual file systems] use the file and directory abstraction to
  make it easy to access other information. For example, the Linux
  \url{/dev} file system is used to access devices through the file
  system, the \url{/sys} file system is used to access features within
  the system kernel, and the \url{/net} file system accesses the
  automounter. Virtual file systems are not typically of interest in
  MEDEX because they do not leave residual information on a storage
  device. However, virtual file systems are relevant in malware
  analysis and intrusion response.
\end{description}

As this chapter is written, specific file systems of interest include:
\begin{description}
\item[FAT12, FAT16 and FAT32] file systems developed by Microsoft for use with
  DOS. FAT refers to the File Allocation Table, an array of integers
  that is used to determine if a cluster is in use or free. (In FAT, a
  \emph{cluster} is a block of 1, 2, 4, 8 or more disk sectors.) The number
  refers to the size of the integers in the array (12 bits, 16 bits or
  32 bits). Today implementations of FAT are built in to practically every
  operating system, including Windows, Linux, MacOS, most digital
  cameras, and practically every other device with a USB or SD
  interface. 
\item[NTFS] 
\item[HFS+]
\item[YAFFS2]
\item[EXT2/3]
\item[EXT4]
\end{description}

Information can be hidden in a file system by storing data in blocks
that are allocated but not used to hold content\cite{dfrws2005:KnutEcksteinAndMarkoJahnke}. 

\section{Exercise: Windows - Creating and Testing FAT32 Disks}

This exercise is designed to help you understand how file systems
store information and the opportunities for retaining
privacy-sensitive information that is not visible to the user. 

This exercise will be done with a \emph{virtual disk}. Like a physical
mass storage device, a virtual disk consists of a set of numbered data
blocks. But instead of storing those blocks directly in on a mass
storage device, the blocks are stored in another file. The operating
system treats the virtual disk like a physical drive: it can be
formatted, files can be copied to it, and so on. But we can also
access the underlying file that holds the virtual disk. This makes it
easier to inspect the individual data blocks.

We will use the Windows |DISKPART| command to create and manage virtual
disks. |DISKPART| is a command-line tool that needs to run as
Administrator.

We will be using these commands:
\begin{tabular}{ll}
help partition & shows partition command available\\
create vdisk & Creates a virtual disk \\
list disk & Shows available disks \\
list vdisk & Shows detailed information about the available VDisks\\
list partition & Shows available partitions \\
\end{tabular}

We will create a text file on the virtual disk that has a sensitive file
name and file contents. We will then take a screen shot and save it on
the computer's hard drive as a JPEG. We will then delete the text file
and empty the trash. Finally we will inspect the disk image and see if
we can find the text file and the screen shot.



\begin{steps}
\step Click the Start button, type |diskpart| into the search field,
and click on the |diskpart| program icon.
\step The User Account Control will ask you to verify that you want to
run the DiskPart, a program that can make ``changes to the computer.''
Click ``Yes.''  The |diskpart.exe| window should appear.
\step If you haven't done so already, right-click on the window's
titlebar and select ``Properties.'' Click on the Layout tab and change
the ``Screen Buffer Size'' to 9999. This will allow you to scroll
backwards and see the entire history of your DISKPART session.
\step Type |help| to see the list of commands that the DISKPART
command supports. You may need to make the window larger.

\step Create a 16MB virtual disk with the command:

\begin{Verbatim}
DISKPART> create vdisk file="C:\disk1.vhd" maximum=16
\end{Verbatim}

\step Attach the disk you just created:
\begin{Verbatim}
DISKPART> attach vdisk
\end{Verbatim}

\step Verify that the disk is attached:
\begin{Verbatim}
DISKPART> list disk
...
DISKPART> list vdisk
\end{Verbatim}

\step Now we need to create a partition, assign a letter to that
partition, and format the drive. We will call this drive K: and format
it with FAT. 
\begin{Verbatim}
DISKPART> create partition primary
DISKPART> assign letter=k
DISKPART> format fs=fat label="WORK"
\end{Verbatim}

At this point we have a 16MB virtual disk attached to the computer as
drive K:. The drive's contents are stored in the file |C:\disk1.vhd|.

Lastly, we want to put a file on this disk. For test purposes we are
going to use a small file with a distinctive file name and file
contents.

\step Run the windows NotePad program by clicking the Start program,
typing ``Notepad'' into the search field, and clicking on the icon.

\step Type this text into the document:  ``The phone number is
202-555-1212.'' You can use your own phone number and add additional
information if you wish.

\step Select ``File/Save As'' and save the file on the WORK (K:) drive
with the name ``file-name-example-0001.txt''. 

\step Close the Notepad program.

\step Now we will detach the virtual disk:

\begin{Verbatim}
DISKPART> detach vdisk
\end{Verbatim}
\end{steps}

\subsection{Looking for the files}
The 

\section{Terminology}
\sgraphic{art/diskpart-1}{The Windows diskpart command is used to create and
  manage physical and virtual disks.}


Allocated

Blocks

Carving

Clusters

Deleted

File Allocation Table

Free List

inode

Hard Link

Symbolic Link

Master File Table

Overwritten

TRIM Command

Undelete

Unlink



\section{File Deletion and Deleted File Recovery}\label{deleted_file_recovery}
Different file system implementations delete files in different ways.

Traditionally, file systems simply \emph{unlinked} files---the pointer
to the file was removed from the directory, and the blocks associated
with the file were returned to the free list.


\section{Exploiting File System Metadata}
File system metadata can be used to determine usage. 
\cite{dfrws2011:JonathanGrier}
